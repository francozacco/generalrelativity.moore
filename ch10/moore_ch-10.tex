\documentclass[11pt]{article}
\usepackage{amssymb}
\usepackage{amsthm}
\usepackage{enumitem}
\usepackage{physics,amsmath}
\usepackage{bm}
\usepackage{adjustbox}
\usepackage{mathrsfs}
\usepackage{graphicx}
\usepackage{siunitx}
\usepackage[mathscr]{euscript}


\title{\textbf{Solved selected problems of General Relativity - Thomas A. Moore}}
\author{Franco Zacco}
\date{}

\addtolength{\topmargin}{-3cm}
\addtolength{\textheight}{3cm}

\newcommand{\hatr}{\bm{\hat{r}}}
\newcommand{\hatn}{\bm{\hat{n}}}
\newcommand{\hatx}{\bm{\hat{x}}}
\newcommand{\haty}{\bm{\hat{y}}}
\newcommand{\hatz}{\bm{\hat{z}}}
\newcommand{\hatth}{\bm{\hat{\theta}}}
\newcommand{\hatphi}{\bm{\hat{\phi}}}
\newcommand{\hatrho}{\bm{\hat{\rho}}}
\newcommand{\er}{\bm{e}_r}
\newcommand{\etht}{\bm{e}_\theta}

\theoremstyle{definition}
\newtheorem*{solution*}{Solution}
\renewcommand*{\proofname}{Solution}

\begin{document}
\maketitle
\thispagestyle{empty}

\section*{Chapter 10 - Particle Orbits}

\begin{proof}{\textbf{BOX 10.1} - Exercise 10.1.1.}
    The $\mu = \theta$ component of the geodesic equation is
    \begin{align*}
        0 = \dv{\tau}(g_{\theta\nu}\dv{x^\nu}{\tau})
        - \frac{1}{2}\partial_\theta g_{\alpha\beta}
        \dv{x^\alpha}{\tau}\dv{x^\beta}{\tau}
    \end{align*}
    Given that the only non-zero component of $g_{\theta\nu}$ is
    $g_{\theta\theta}$ and that the only $\theta$-dependent component is
    $g_{\phi\phi}$ we have theta
    \begin{align*}
        0 = \dv{\tau}(g_{\theta\theta}\dv{\theta}{\tau})
        - \frac{1}{2}\partial_\theta g_{\phi\phi}
        \dv{\phi}{\tau}\dv{\phi}{\tau}
    \end{align*}
    Hence
    \begin{align*}
        0 &= \dv{\tau}(r^2\dv{\theta}{\tau})
        - r^2\sin\theta\cos\theta \bigg(\dv{\phi}{\tau}\bigg)^2\\
        0 &= r^2\bigg(\dv{\theta}{\tau}\bigg)^2
        + 2r\dv{r}{\tau}\dv{\theta}{\tau}
        - r^2\sin\theta\cos\theta \bigg(\dv{\phi}{\tau}\bigg)^2
    \end{align*}
    Finally if $\theta = \pi/2$ then $\dv[2]{\theta}{\tau} = 0$,
    $\dv{\theta}{\tau} = 0$ and $\cos\theta = 0$ therefore $\theta = \pi/2$
    is a solution to the equation.
\end{proof}
\cleardoublepage
\begin{proof}{\textbf{BOX 10.2} - Exercise 10.2.1.}
    We want to verify equation (10.22) using that
    \begin{align*}
        \bigg(1 - \frac{2GM}{r}\bigg)\dv{t}{\tau} = e
        \qquad r^2\sin^2\theta\dv{\phi}{\tau} = l
    \end{align*}
    Hence considering orbits in the equatorial plane $\theta = \pi/2$ we get
    that
    \begin{align*}
        -1 &= g_{tt}\left(\dv{t}{\tau}\right)^2 + g_{rr}\left(\dv{r}{\tau}\right)^2
        + g_{\theta\theta}\left(\dv{\theta}{\tau}\right)^2
        + g_{\phi\phi}\left(\dv{\phi}{\tau}\right)^2\\
        -1 &= -\bigg(1 - \frac{2GM}{r}\bigg)\left(\dv{t}{\tau}\right)^2
        + \bigg(1 - \frac{2GM}{r}\bigg)^{-1}\left(\dv{r}{\tau}\right)^2
        + r^2\left(\dv{\phi}{\tau}\right)^2\\
        1 &= \bigg(1 - \frac{2GM}{r}\bigg)^{-1}e^2
        - \bigg(1 - \frac{2GM}{r}\bigg)^{-1}\left(\dv{r}{\tau}\right)^2
        - \frac{l^2}{r^2}
    \end{align*}
    Where we used that $d\theta/d\tau = 0$.
\end{proof}
\begin{proof}{\textbf{BOX 10.2} - Exercise 10.2.2.}
    From the equation
    \begin{align*}
        1 &= \bigg(1 - \frac{2GM}{r}\bigg)^{-1}e^2
        - \bigg(1 - \frac{2GM}{r}\bigg)^{-1}\left(\dv{r}{\tau}\right)^2
        - \frac{l^2}{r^2}
    \end{align*}
    We obtain equation (10.8) as follows
    \begin{align*}
        1 &= \bigg(1 - \frac{2GM}{r}\bigg)^{-1}e^2
        - \bigg(1 - \frac{2GM}{r}\bigg)^{-1}\left(\dv{r}{\tau}\right)^2
        - \frac{l^2}{r^2}\\
        \bigg(1 - \frac{2GM}{r}\bigg) &= e^2
        - \left(\dv{r}{\tau}\right)^2
        - \bigg(1 - \frac{2GM}{r}\bigg)\frac{l^2}{r^2}\\
        1 - e^2 &= \frac{2GM}{r}
        - \left(\dv{r}{\tau}\right)^2
        - \frac{l^2}{r^2} + \frac{2GMl^2}{r^3}\\
        \frac{1}{2}(e^2 - 1) &=
        \frac{1}{2}\left(\dv{r}{\tau}\right)^2
        - \frac{GM}{r}
        + \frac{l^2}{2r^2} - \frac{GMl^2}{r^3}
    \end{align*}
\end{proof}
\cleardoublepage
\begin{proof}{\textbf{BOX 10.3} - Exercise 10.3.1.}
    Newtonian Energy is defined as
    \begin{align*}
        E = \frac{1}{2}m\left[\left(\dv{r}{t}\right)^2 + r^2\left(\dv{\phi}{t}\right)^2\right]
        - \frac{GMm}{r}
    \end{align*}
    So dividing the equation by $m$ and defining $\tilde{E}_N = E/m$ we get
    that
    \begin{align*}
        \tilde{E}_N = \frac{1}{2}
        \left[\left(\dv{r}{t}\right)^2 + r^2\left(\dv{\phi}{t}\right)^2\right]
        - \frac{GM}{r}
    \end{align*}
    Also, using $l = r^2 d\phi/dt$ then $r^2(d\phi/dt)^2 = l^2/r^2$ hence
    \begin{align*}
        \tilde{E}_N = \frac{1}{2}\left(\dv{r}{t}\right)^2 + \frac{l^2}{2r^2}
        - \frac{GM}{r}
    \end{align*}
    Finally defining
    \begin{align*}
        \tilde{V}_N(r) = \frac{l^2}{2r^2} - \frac{GM}{r}
    \end{align*}
    We get that
    \begin{align*}
        \tilde{E}_N = \frac{1}{2}\left(\dv{r}{t}\right)^2 + \tilde{V}_N(r)
    \end{align*}
    Which is equation (10.10).
\end{proof}
\cleardoublepage
\begin{proof}{\textbf{BOX 10.4} - Exercise 10.4.1}\\
    Equation (10.9) states that
   \begin{align*}
        \tilde{V}(r) = -\frac{GM}{r} + \frac{l^2}{2r^2} - \frac{GMl^2}{r^3} 
   \end{align*}
   Hence derivating with respect to $r$ and setting it equal to zero gives us
   \begin{align*}
    0 = \frac{GM}{r^2} - \frac{l^2}{r^3} + \frac{3GMl^2}{r^4} 
    \end{align*}
\end{proof}
\begin{proof}{\textbf{BOX 10.4} - Exercise 10.4.2}\\
    Let $u_c = 1/r_c$ then equation (10.26) becomes
    \begin{align*}
        0 &= u_c^2GM - u_c^3l^2 + 3u_c^4GMl^2\\
        0 &= GM - u_cl^2 + 3u_c^2GMl^2
    \end{align*}
    Where we divided by $u_c^2$. Then solving this quadratic equation we get
    that
    \begin{align*}
        u_c &= \frac{l^2 \pm \sqrt{l^4 - 12(GMl)^2}}{6GMl^2}\\
        u_c &= \frac{1 \pm \sqrt{1 - 12(GM/l)^2}}{6GM}
    \end{align*}
    Now we replace $u_c = 1/r_c$ again and we solve for $r_c$ as follows
    \begin{align*}
        \frac{1}{r_c} &= \frac{1 \pm \sqrt{1 - 12(GM/l)^2}}{6GM}\\
        r_c &= \frac{6GM}{1 \pm \sqrt{1 - 12(GM/l)^2}}
    \end{align*}
\end{proof}
\begin{proof}{\textbf{BOX 10.4} - Exercise 10.4.3}\\
    We know that the square root must be real so we can get the smallest
    value $l$ can have from there, hence
    \begin{align*}
        1 - 12\left(\frac{GM}{l}\right)^2 &\geq 0\\
        \left(\frac{GM}{l}\right)^2 &\leq \frac{1}{12}\\
        \left(\frac{1}{l^2}\right) &\leq \frac{1}{12(GM)^2}\\
        l^2 &\geq 12(GM)^2\\
        l &\geq \sqrt{12}GM
    \end{align*}
    Therefore the smallest value $l$ can have is $\sqrt{12}GM$.
\end{proof}
\cleardoublepage
\begin{proof}{\textbf{BOX 10.5} - Exercise 10.5.1}
    We know that
    \begin{align*}
        0 &= \dv{\tau}(g_{rr}\dv{r}{\tau})
        - \frac{1}{2}\pdv{g_{\alpha\beta}}{r}\dv{x^\alpha}{\tau}\dv{x^\beta}{\tau}\\
        0 &= \dv{\tau}(g_{rr}\dv{r}{\tau}) - \frac{1}{2}\left(
            \pdv{g_{tt}}{r}\dv{t}{\tau}\dv{t}{\tau}
            + \pdv{g_{rr}}{r}\dv{r}{\tau}\dv{r}{\tau}
            + \pdv{g_{\theta}}{r}\dv{\theta}{\tau}\dv{\theta}{\tau}
            + \pdv{g_{\phi}}{r}\dv{\phi}{\tau}\dv{\phi}{\tau}
        \right)
    \end{align*}
    If we let $dr/d\tau = 0$ and $d\theta/d\tau = 0$ we get that
    \begin{align*}
        0 &= - \frac{1}{2}\left(
            \pdv{g_{tt}}{r}\dv{t}{\tau}\dv{t}{\tau}
            + \pdv{g_{\phi}}{r}\dv{\phi}{\tau}\dv{\phi}{\tau}
        \right)\\
        0 &= \pdv{g_{tt}}{r}\left(\dv{t}{\tau}\right)^2
        + \pdv{g_{\phi}}{r}\left(\dv{\phi}{\tau}\right)^2
    \end{align*}
    Which is the equation we were looking for.
\end{proof}
\begin{proof}{\textbf{BOX 10.5} - Exercise 10.5.2}
    We know that 
    \begin{align*}
        0 &= \pdv{g_{tt}}{r} + \pdv{g_{\phi\phi}}{r}\Omega^2
    \end{align*}
    Replacing the values of $g_{tt}$, $g_{\phi\phi}$ and considering
    that $\theta =\pi/2$ we get that
    \begin{align*}
        0 &= -\pdv{r}(1-\frac{2GM}{r}) + \pdv{r}(r^2)\Omega^2\\
        0 &= -\frac{2GM}{r^2} + 2r\Omega^2\\
        \Omega^2 &= \frac{GM}{r^3}
    \end{align*}
    Finally since $T = |2\pi/\Omega|$ we have that
    \begin{align*}
        T^2 = \frac{4\pi^2}{\Omega^2} &= \frac{4\pi^2}{\frac{GM}{r^3}}
        = \frac{4\pi^2}{GM}r^3
    \end{align*}
\end{proof}
\cleardoublepage
\begin{proof}{\textbf{BOX 10.6} - Exercise 10.6.1}\\
    Setting the first expression equal to 0 and multiplying by $2/r$ we get
    that
    \begin{align*}
        0 &= \frac{2GM}{r^3} - \frac{2l^2}{r^4} + \frac{6GMl^2}{r^5}
    \end{align*}
    Now we add it the the second expression to get
    \begin{align*}
        \dv[2]{\tilde{V}}{r}
        &= \frac{l^2}{r^4} - \frac{6GMl^2}{r^5}
    \end{align*}
    But we know that $\dv[2]{\tilde{V}}{r} > 0$ hence
    \begin{align*}
        \frac{l^2}{r^4} - \frac{6GMl^2}{r^5} &> 0\\
        l^2 &> \frac{6GMl^2}{r}\\
        r &>\frac{6GMl^2}{l^2}
    \end{align*}
    Therefore the minima can only exist for $r > 6GM$.
\end{proof}
\cleardoublepage
\begin{proof}{\textbf{BOX 10.7} - Exercise 10.7.1}\\
    Let $l/GM = \sqrt{12}$, $r = 6GM$ and $dr/d\tau = 0$ then from equation 
    10.8 we have that
    \begin{align*}
        \frac{1}{2}(e^2 - 1) &= -\frac{1}{6} + \frac{12}{72} - \frac{12}{216}\\
        e^2 - 1 &= - \frac{1}{9}\\
        e &= \sqrt{\frac{8}{9}}
    \end{align*}
    Now, let us subtract the value $e=1$ for the particle when it was at
    infinity to find the change in $e$ during the inspiral process
    \begin{align*}
        \Delta e = \sqrt{\frac{8}{9}} - 1 = -0.057
    \end{align*}
\end{proof}
\cleardoublepage
\begin{proof}{\textbf{P10.1}}
    Let us approximate the area of the disk between $R = 6GM$ and $2R$ as
    \begin{align*}
        A = \pi(4R^2 - R^2) = 3\pi R^2 = 108\pi (GM)^2   
    \end{align*}
    Then from equation (10.17) we have that
    \begin{align*}
        L &= \sigma A T^4
    \end{align*}
    Hence
    \begin{align*}
        T^4 &= \frac{L}{\sigma (108\pi (GM)^2)}\\
        T^4 &= \frac{L}{L_\odot}\frac{M_\odot^2}{M^2}
        \frac{L_\odot}{\sigma 108\pi G^2M_\odot^2}\\
        T^4 &= \frac{L}{L_\odot}\frac{M_\odot^2}{M^2}
        \frac{3.9 \times 10^{26}}{(5.67\times 10^{-8})108\pi(1477)^2}\\
        T &= \left(\frac{L}{L_\odot}\right)^{1/4}
        \left(\frac{M_\odot}{M}\right)^{1/2}
        1.74 \times 10^6~K\\
        T &\approx \left(\frac{L}{L_\odot}\right)^{1/4}
        \left(\frac{M_\odot}{M}\right)^{1/2} 2 \times 10^6~K
    \end{align*}
\end{proof}
\cleardoublepage
\begin{proof}{\textbf{P10.2}}
    An object falling radially inward towards a black hole starting at rest
    at infinity will have $\tilde{E} = 0$ hence
    \begin{align*}
        \frac{1}{2}\bigg(\dv{r}{\tau}\bigg)^2 - \frac{GM}{r} + \frac{l^2}{2r^2}
        - \frac{GMl^2}{r^3} &= 0
    \end{align*}
    But for a radially falling object we have that $l = 0$ then
    \begin{align*}
        \frac{1}{2}\bigg(\dv{r}{\tau}\bigg)^2 &= \frac{GM}{r}\\
        \dv{r}{\tau} &= \sqrt{\frac{2GM}{r}}\\
        \int_0^\tau d\tau &= \int^{10GM}_{2GM} \sqrt{\frac{r}{2GM}} dr\\
        \tau &= \frac{1}{\sqrt{2GM}}
        \bigg[\frac{2}{3}r^{2/3}\bigg]^{10GM}_{2GM}\\
        \tau &= \frac{2}{3\sqrt{2GM}}\bigg[r^{3/2}\bigg]^{10GM}_{2GM}\\
        \tau &= \frac{2(GM)^{3/2}}{3\sqrt{2GM}}\bigg[10^{3/2} - 2^{3/2}\bigg]\\
        \tau &= \frac{\sqrt{2}GM}{3}\bigg[10\sqrt{10}-2\sqrt{2}\bigg]\\
        \tau &\approx 13.6GM
    \end{align*}
\end{proof}
\cleardoublepage
\begin{proof}{\textbf{P10.3}}
    For an observer at rest we know that $u^t$ is the only non-zero component
    then the product $-\bm{p} \cdot \bm{u}_{obs}$ becomes
    \begin{align*}
        -\bm{p} \cdot \bm{u}_{obs} = g_{tt}p^t u^t_{obs}
    \end{align*}
    From equation 9.20 we know that
    \begin{align*}
        u^t_{obs} = \bigg(1 - \frac{2GM}{r}\bigg)^{-1/2}
    \end{align*}
    Also, we know that for the falling object
    \begin{align*}
        \bigg(1 - \frac{2GM}{r}\bigg)u^t = e
    \end{align*}
    Hence 
    \begin{align*}
        p^t = m u^t = me\bigg(1 - \frac{2GM}{r}\bigg)^{-1}
    \end{align*}
    Then
    \begin{align*}
        E &= -\bm{p} \cdot \bm{u}_{obs}\\
        &=\bigg(1 - \frac{2GM}{r}\bigg)m e\bigg(1 - \frac{2GM}{r}\bigg)^{-1}
        \bigg(1 - \frac{2GM}{r}\bigg)^{-1/2}\\
        &= m e\bigg(1 - \frac{2GM}{r}\bigg)^{-1/2}
    \end{align*}
    But if the observer is at $r = 6GM$ we get that
    \begin{align*}
        E &= m e\bigg(1 - \frac{2GM}{6GM}\bigg)^{-1/2} = 1.2247~me
    \end{align*}
    On the other hand, the speed the observer will measure can be computed from
    the following formula
    \begin{align*}
        E &= \frac{m}{\sqrt{1-v^2}}\\
        v &= \sqrt{1 -\frac{m^2}{E^2}}
    \end{align*}
    Then for an object falling with $e = 1$ we have that
    \begin{align*}
        v &= \sqrt{1 -\frac{1}{(1.2247)^2}} = 0.5773
    \end{align*}
    and for an object where $e = 2$ we have that
    \begin{align*}
        v &= \sqrt{1 -\frac{1}{4(1.2247)^2}} = 0.9128
    \end{align*}
    Therefore the object moving with $e=2$ will be
    $0.9128/0.5773 \approx 1.5811$ faster than the one moving with $e=1$.
\end{proof}
\cleardoublepage
\begin{proof}{\textbf{P10.4}}
\begin{itemize}
    \item [\textbf{a.}] From BOX 10.2 we know that
    \begin{align*}
        \bigg(1 - \frac{2GM}{r}\bigg)^{-1}\left(\dv{r}{\tau}\right)^2
        &= \bigg(1 - \frac{2GM}{r}\bigg)^{-1}e^2 - \frac{l^2}{r^2} - 1
    \end{align*}
    Hence
    \begin{align*}
        \left(\dv{r}{\tau}\right)^2
        &= e^2 - \bigg(1 - \frac{2GM}{r}\bigg)\frac{l^2}{r^2}
        - \bigg(1 - \frac{2GM}{r}\bigg)\\
        \dv{r}{t}\dv{t}{\tau}
        &= \sqrt{e^2 - \bigg(1 - \frac{2GM}{r}\bigg)\frac{l^2}{r^2}
        - \bigg(1 - \frac{2GM}{r}\bigg)}\\
        \dv{r}{t}e\bigg(1 - \frac{2GM}{r}\bigg)^{-1}
        &= \sqrt{e^2 - \bigg(1 - \frac{2GM}{r}\bigg)\frac{l^2}{r^2}
        - \bigg(1 - \frac{2GM}{r}\bigg)}\\
        \dv{r}{t}
        &= \bigg(1 - \frac{2GM}{r}\bigg)\sqrt{1 - \bigg(1 - \frac{2GM}{r}\bigg)
        \frac{l^2}{r^2e^2}
        - \bigg(1 - \frac{2GM}{r}\bigg)\frac{1}{e^2}}
    \end{align*}
    Where we used the value we have for $dt/d\tau$.
    Then as $r$ approaches to $2GM$ we see that $dr/dt$ approaches to 0.
    \item [\textbf{b.}]
    If we drop an object from rest in the equatorial plane at radial
    coordinate $r_0$, then $e$ and $l$ become
    \begin{align*}
        l &= 0 \qquad e^2 = 1 - \frac{2GM}{r_0} 
    \end{align*}
    Hence for $dr/dt$ we have that
    \begin{align*}
        \dv{r}{t}
        &= \bigg(1 - \frac{2GM}{r}\bigg)\sqrt{
        1 - \bigg(1 - \frac{2GM}{r}\bigg)\bigg(1 - \frac{2GM}{r_0}\bigg)^{-1}}\\
        &= \bigg(1 - \frac{2GM}{r}\bigg)\sqrt{
        \bigg(1 - \frac{2GM}{r_0}\bigg)^{-1}\bigg(
            \bigg(1 - \frac{2GM}{r_0}\bigg)- \bigg(1 - \frac{2GM}{r}\bigg)
        \bigg)}\\
        &= \bigg(1 - \frac{2GM}{r}\bigg)
        \bigg(1 - \frac{2GM}{r_0}\bigg)^{-1/2}
        \sqrt{\frac{2GM}{r} - \frac{2GM}{r_0}}
    \end{align*}
    And for $dr/d\tau$ we get that
    \begin{align*}
        \dv{r}{\tau}
        &= \sqrt{e^2 - \bigg(1 - \frac{2GM}{r}\bigg)}\\
        &= \sqrt{\bigg(1 - \frac{2GM}{r_0}\bigg) - \bigg(1 - \frac{2GM}{r}\bigg)}\\
        &= \sqrt{\frac{2GM}{r} - \frac{2GM}{r_0}}
    \end{align*}
\end{itemize}
\end{proof}
\cleardoublepage
\begin{proof}{\textbf{P10.5}}
    Let us assume the proper time measured by the object is the same as the
    object goes from $r_0$ to $r_1$ and as it falls from $r_1$ to $r_0$.
    When the object falls from $r_1$ to $r_0$ it does so from rest and 
    radially so we can assume that $l = 0$ and at $r = r_1$ the object comes
    to rest then $dr_1/d\tau = 0$ at this moment hence
    \begin{align*}
        \frac{1}{2}(e^2 - 1) &= -\frac{GM}{r_1}\\
        e &= \sqrt{1 - \frac{2GM}{r_1}}
    \end{align*}
    Then using the equation (10.8) replacing the value we have for $e$ we get
    that
    \begin{align*}
        \frac{1}{2}\left(\left(1 - \frac{2GM}{r_1}\right) - 1\right)
        &= \frac{1}{2}\left(\dv{r}{\tau}\right)^2 - \frac{GM}{r}\\
        - \frac{2GM}{r_1}
        &= \left(\dv{r}{\tau}\right)^2 - \frac{2GM}{r}\\
        \dv{r}{\tau} &= \sqrt{\frac{2GM}{r} - \frac{2GM}{r_1}}
    \end{align*}
    Hence by integration and changing variables to $u = r/r_1$ we have that
    \begin{align*}
        \int_0^\tau d\tau
        &= \int_{r_0}^{r_1} \left(\frac{2GM}{r} - \frac{2GM}{r_1}\right)^{-1/2}~dr\\
        \tau
        &= \int_{r_0/r_1}^{1} \left(\frac{2GM}{ur_1} - \frac{2GM}{r_1}\right)^{-1/2}~r_1du\\
        \tau &= r_1\left(\frac{2GM}{r_1}\right)^{-1/2} \int_{r_0/r_1}^{1}
        \left(\frac{1}{u} - 1\right)^{-1/2}~du\\
        \tau &= r_1\left(\frac{2GM}{r_1}\right)^{-1/2}
        \left[-\arctan\sqrt{\frac{1}{u} - 1} - u\sqrt{\frac{1}{u} - 1}
        \right]_{r_0/r_1}^{1}\\
        \tau &= r_1\left(\frac{2GM}{r_1}\right)^{-1/2}
        \left[\arctan\sqrt{\frac{r_1}{r_0} - 1} + \frac{r_0}{r_1}\sqrt{\frac{r_1}{r_0} - 1}
        \right]
    \end{align*}
    Finally since this is the proper time to go from $r_0$ to $r_1$ we multiply
    by 2 to get the total proper time of the round trip, then
    \begin{align*}
        \tau_{total} &= 2r_1\left(\frac{2GM}{r_1}\right)^{-1/2}
        \left[ \arctan\sqrt{\frac{r_1}{r_0} - 1}
        + \frac{r_0}{r_1}\sqrt{\frac{r_1}{r_0} - 1}\right]
    \end{align*}
\end{proof}
\cleardoublepage
\begin{proof}{\textbf{P10.6}}
\begin{itemize}
    \item [\textbf{a.}] Given that the object is in a stable circular orbit we
    can compute the radius of the orbit as follows
    \begin{align*}
        r &= \frac{6GM}{1 - \sqrt{1 - 12(GM/l)^2}}\\ 
        &= \frac{6\cdot 2.2}{1 - \sqrt{1 - 12(2.2/13.2)^2}}\\
        &= \frac{13.2}{1 - 0.816}\\
        &= 71.93~km
    \end{align*}
    Where we used the negative result of the square root because it gives us
    the stable orbit radius.
    \item [\textbf{b.}] From equation (10.7) assuming the circular orbit is
    in the equatorial plane we can compute $d\phi/d\tau$ as follows
    \begin{align*}
        \dv{\phi}{\tau} = \frac{l}{r^2} = 0.00255
    \end{align*}
    Now by integration we have that
    \begin{align*}
        \int_0^\tau d\tau &= \frac{1}{0.00255} \int_0^{2\pi} d\phi\\
        \tau &= \frac{2\pi}{0.00255}\\
        \tau &= 2463.99 ~km
    \end{align*}
    But since $299.8~km = 1~ms$ the period of orbit is
    \begin{align*}
        \tau &= 8.21~ms
    \end{align*}
    \item [\textbf{c.}] From Box 10.5 we know an observer at infinity
    measures the period of an object's circular orbit as follows
    \begin{align*}
         T = \sqrt{\frac{4\pi^2}{GM} r^3} = \sqrt{\frac{4\pi^2}{2.2} \cdot 71.93^3}
         = 2584.24~km = 8.62~ms 
    \end{align*}
\end{itemize}
\end{proof}
\cleardoublepage
\begin{proof}{\textbf{P10.7}}
\begin{itemize}
    \item [\textbf{a.}] From equation 10.11 we know that
    \begin{align*}
        r_c = \frac{6GM}{1 \pm \sqrt{1 - 12(GM/l)^2}}
    \end{align*}
    Then 
    \begin{align*}
        \pm \sqrt{1 - 12(GM/l)^2} &= \frac{6GM}{r_c} - 1\\
        1 - 12(GM/l)^2 &= \left(\frac{6GM - r_c}{r_c}\right)^2\\
        (1/l)^2 &= \frac{1}{12(GM)^2}\left(1 - \frac{(6GM - r_c)^2}{r_c^2}\right)\\
        l^2 &= 12(GM)^2\left(\frac{r_c^2 - 36(GM)^2 +12GMr_c - r_c^2}{r_c^2}\right)^{-1}\\
        l^2 &= 12(GM)^2\frac{r_c^2}{12GM(r_c - 3GM)}\\
        l^2 &= \frac{GMr_c^2}{r_c - 3GM}
    \end{align*}
    \item [\textbf{b.}] From equation 10.8 we know that
    \begin{align*}
        \tilde{E} = \frac{1}{2}\left(\dv{r}{\tau}\right)^2
        - \frac{GM}{r} + \frac{l^2}{2r^2} - \frac{GMl^2}{r^3}
    \end{align*}
    But for a circular orbit we replace $r$ to $r_c$ and we drop the term
    $dr/d\tau$, also, we replace the value of $l^2$ we got as follows
    \begin{align*}
        \tilde{E} &=
        - \frac{GM}{r_c} + \frac{1}{2r_c^2}\frac{GMr_c^2}{r_c - 3GM}
        - \frac{GM}{r_c^3}\frac{GMr_c^2}{r_c - 3GM}\\
        &= - \frac{GM}{r_c} + \frac{GM}{2(r_c - 3GM)}
        - \frac{(GM)^2}{r_c(r_c - 3GM)}\\
        &= - \frac{GM}{r_c} + \frac{GM}{2r_c}\frac{1}{1 - \frac{3GM}{r_c}}
        - \frac{GM}{r_c}\frac{GM}{r_c - 3GM}\\
        &= - \frac{GM}{r_c} - \frac{(GM)^2}{r_c^2}\left(1 - \frac{3GM}{r_c}\right)^{-1}
        + \frac{GM}{2r_c}\left(1 - \frac{3GM}{r_c}\right)^{-1}\\
        &= - \frac{GM}{2r_c}\left(1 - \frac{3GM}{r_c}\right)^{-1}
        \left[2\left(1 - \frac{3GM}{r_c}\right) +\frac{2GM}{r_c} -1\right]\\
        &= - \frac{GM}{2r_c}\left(1 - \frac{3GM}{r_c}\right)^{-1}
        \left[1 - \frac{4GM}{r_c}\right]
    \end{align*}
\cleardoublepage
    \item [\textbf{c.}] From equation 10.8 and the result from part (b) we
    have that
    \begin{align*}
        \frac{1}{2}(e^2 -1 ) = - \frac{GM}{2r_c}\left(1 - \frac{3GM}{r_c}\right)^{-1}
        \left[1 - \frac{4GM}{r_c}\right]
    \end{align*}
    Hence
    \begin{align*}
        e = \sqrt{1 - \frac{GM}{r_c}\left(1 - \frac{3GM}{r_c}\right)^{-1}
        \left[1 - \frac{4GM}{r_c}\right]}
    \end{align*}
    Replacing $r_c = 6GM$ we get that
    \begin{align*}
        e &= \sqrt{1 - \frac{GM}{6GM}\left(1 - \frac{3GM}{6GM}\right)^{-1}
        \left[1 - \frac{4GM}{6GM}\right]}\\
        &= \sqrt{1 - \frac{1}{6}\left(1 - \frac{1}{2}\right)^{-1}
        \left[1 - \frac{2}{3}\right]}\\
        &= \sqrt{1 - \frac{1}{9}}\\
        &= \sqrt{\frac{8}{9}}
    \end{align*}
\end{itemize}
\end{proof}
\begin{proof}{\textbf{P10.8}}
    If an object starts at rest at infinity with an angular-momentum-per-unit-mass
    $l$ the equation (10.8) becomes
    \begin{align*}
        0 = - \frac{GM}{r} + \frac{l^2}{2r^2} - \frac{GMl^2}{r^3}
    \end{align*}
    since $e = 1$ and $dr/d\tau = 0$. Hence we can write it as
    \begin{align*}
        0 &= - GMr^2 + \frac{l^2}{2}r - GMl^2
    \end{align*}
    And the solutions to this quadratic equation are
    \begin{align*}
        r &= \frac{-l^2/2 \pm \sqrt{l^4/4 - 4G^2M^2l^2}}{-2GM}\\
        r &= \frac{-l^2/2 \pm \sqrt{l^4/4(1 - 16G^2M^2/l^2)}}{-2GM}\\
        r &= \frac{-l^2(1 \mp \sqrt{1 - 16G^2M^2/l^2})}{-4GM}\\
        r &= \frac{l^2(1 \mp \sqrt{1 - 16G^2M^2/l^2})}{4GM}
    \end{align*}
    Hence we see that if $l = 4GM$ we get that 
    \begin{align*}
        r = 4GM
    \end{align*}
    Which is an unstable orbit since $r < 6GM$ therefore the particle can
    spiral into an unstable circular orbit at $r_c = 4GM$.
\end{proof}
\cleardoublepage
\begin{proof}{\textbf{P10.9}}
    Let a spaceship be in a stable circular orbit at $r = 10GM$ around a
    supermassive black hole whose mass is $10^6$ solar masses.
    \begin{itemize}
        \item [\textbf{a.}] The physical distance around the orbit or the
        circumference of the orbit can be found by integrating the spacetime
        interval $ds = rd\phi$ as follows
        \begin{align*}
            \text{circumference} = \oint ds
            &= \int_0^{2\pi} rd\phi = 2\pi r = 20\pi GM
        \end{align*}
        Hence 
        \begin{align*}
            \text{circumference}
            &= 20\pi \cdot 1477 \cdot 10^6\\
            &= 92802646987~m\\
            &= 92802646.987~km
        \end{align*}
        \item [\textbf{b.}] From the results of problem P10.7 we know that for
        an object in a stable circular orbit $\tilde{E}$ is given by
        \begin{align*}
            \tilde{E} &= - \frac{GM}{2r_c}\left(1 - \frac{3GM}{r_c}\right)^{-1}
            \left[1 - \frac{4GM}{r_c}\right]\\
            &= - \frac{1}{20}\left(1 - \frac{3}{10}\right)^{-1}
            \left[1 - \frac{4}{10}\right]\\
            &= - \frac{1}{20} \cdot 1.4285 \cdot 0.6\\
            &= -0.042855
        \end{align*}
        Also, we know that $l$ is given by
        \begin{align*}
            l &= \sqrt{\frac{GMr_c^2}{r_c - 3GM}}\\
            &= \sqrt{\frac{100(GM)^3}{7GM}}\\
            &= 3.7796 GM\\
            &= 5582469200~m
        \end{align*}
\cleardoublepage
        \item [\textbf{c.}] From equation (10.7) assuming the circular orbit is
        in the equatorial plane we can compute $d\phi/d\tau$ as follows
        \begin{align*}
            \dv{\phi}{\tau} = \frac{l}{r^2} = \frac{3.7796}{100 GM}
            = 2.5589 \times 10^{-11}~m^{-1}
        \end{align*}
        Now by integration we have that
        \begin{align*}
            \int_0^\tau d\tau &= \frac{1}{2.5589 \times 10^{-11}} \int_0^{2\pi} d\phi\\
            \tau &= \frac{2\pi}{2.5589 \times 10^{-11}}\\
            \tau &= 245535630720.29~m = 245535630.72029~km
        \end{align*}
        But since $17.99 \times 10^6 km = 1~min$ the period of orbit is
        \begin{align*}
            \tau &= 13.648~min
        \end{align*}
    \end{itemize}
\end{proof}
\begin{proof}{\textbf{P10.10}}
    From equation (10.7) assuming a circular orbit in the equatorial plane
    we know that
    \begin{align*}
        \omega = \dv{\phi}{\tau} = \frac{l}{r^2}
    \end{align*}
    But also we know that for a circular orbit
    \begin{align*}
        l &= \sqrt{\frac{GMr^2}{r - 3GM}} = \sqrt{\frac{GMr}{1 - 3GM/r}}
    \end{align*}
    Hence
    \begin{align*}
        \omega &= \sqrt{\frac{1}{r^4}\frac{GMr}{1 - 3GM/r}}
        = \sqrt{\frac{GM}{r^3}\frac{1}{1 - 3GM/r}}
    \end{align*}
    Then the ratio $\omega/\Omega$ gives us
    \begin{align*}
        \frac{\omega}{\Omega}
        = \frac{\sqrt{\frac{GM}{r^3}\frac{1}{1 - 3GM/r}}}{\sqrt{\frac{GM}{r^3}}}
        = \left(1 - \frac{3GM}{r}\right)^{-1/2}
    \end{align*}
\end{proof}
\cleardoublepage
\begin{proof}{\textbf{P10.11}}
    We know Newtonian potential energy is given by
    \begin{align*}
        \tilde{V}_N(r) = - \frac{GM}{r} + \frac{l^2}{2r^2}
    \end{align*}
    The radii $r_c$ of possible circular orbits correspond to values of the
    radial coordinate $r$ where $d\tilde{V}_N/dr = 0$ then setting the derivative
    of the equation for $\tilde{V}_N$ equal to zero gives us
    \begin{align*}
        0 = \frac{GM}{r^2} - \frac{l^2}{r^3}
    \end{align*}
    Hence the equation analogous to (10.11) for Newtonian mechanics is
    \begin{align*}
        \frac{GM}{r_c^2} &= \frac{l^2}{r_c^3}\\
        GM &= \frac{l^2}{r_c}\\
        r_c &= \frac{l^2}{GM}
    \end{align*}
    Now, if we consider a large $l$ we can use the binomial approximation
    on equation (10.11) for the stable orbit as follows
    \begin{align*}
        r_c &= \frac{6GM}{1 - (1 - 6(GM)^2/l^2)}\\
        &= \frac{6GM}{6(GM)^2/l^2}\\
        &= \frac{1}{GM/l^2}\\
        &= \frac{l^2}{GM}
    \end{align*}
    Which is the result we got for Newtonian mechanics.
\end{proof}
\cleardoublepage
\begin{proof}{\textbf{P10.12}}
    Let an observer in a spaceship in a circular orbit of radius $r$. We know
    that for a circular trajectory on the equatorial plane 
    \begin{align*}
        l = r^2\bigg(\frac{d\phi}{d\tau}\bigg)
    \end{align*}
    So by integration we get the period this observer experiences as follows
    \begin{align*}
        \int_0^\tau d\tau &= \frac{r^2}{l}\int_0^{2\pi}d\phi\\
        \tau &= \frac{2\pi r^2}{l}
    \end{align*}
    Replacing the value of $l$ from equation (10.32) we get that
    \begin{align*}
        \tau &= \frac{2\pi r^2}{\sqrt{GMr^2/(r - 3GM)}}\\
        \tau &= 2\pi r\sqrt{\frac{(r - 3GM)}{GM}}
    \end{align*}
    Thus the speed for this observer is
    $$v_\tau = \frac{2\pi r}{\tau} = \sqrt{\frac{GM}{(r - 3GM)}}$$
    For an observer at infinity we know the period measured is
    \begin{align*}
        T_\infty = \sqrt{\frac{4\pi^2}{GM}r^3} = 2\pi r \sqrt{\frac{r}{GM}}
    \end{align*}
    and hence the speed for this observer is
    $$v_\infty = \frac{2\pi r}{T_\infty} = \sqrt{\frac{GM}{r}}$$

    On the other hand, we know that a clock at rest at $r$ measures a time
    $T_r$ between two events and this is related to the time $T_\infty$
    a clock at $r = \infty$ measures by the following equation
    \begin{align*}
        T_r = \sqrt{1 - \frac{2GM}{r}} T_\infty
    \end{align*}
    Hence
    \begin{align*}
        T_r = 2\pi r \sqrt{\frac{r}{GM}\bigg(1 - \frac{2GM}{r}\bigg)} 
    \end{align*}
    Thus in this case the speed is
    \begin{align*}
        v_r = \frac{2\pi r}{T_r}
        =\sqrt{\frac{GM}{}}\bigg(1 - \frac{2GM}{r}\bigg)^{-1/2}
    \end{align*}
    We see that $v_\tau > v_r > v_\infty$. The speeds can be bigger than 1
    depending on the values of $GM$ and $r$ but these are not actual speeds.
\end{proof}
\cleardoublepage
\begin{proof}{\textbf{P10.13}}
\begin{itemize}
    \item [\textbf{a.}] From Schwarzschild metric equation where $dr = 0$ and
    $d\theta = 0$ but also since along a photon worldline $ds = 0$ we have that
    \begin{align*}
        0 &= -\left(1 - \frac{2GM}{r}\right)dt^2 + r^2d\phi^2\\
        \frac{r^2d\phi^2}{dt^2} &= \left(1 - \frac{2GM}{r}\right)\\
        V =\frac{rd\phi}{dt} &= \sqrt{1 - \frac{2GM}{r}}
    \end{align*}
    Where $rd\phi/dt$ is the photon's speed as measured by an observer at
    infinity. Considering that photons can orbit a Schwarzschild black hole at
    a constant radial coordinate of $r = 3GM$ we get that
    \begin{align*}
        V &= \sqrt{1 - \frac{2}{3}} = 0.577
    \end{align*}
    \item [\textbf{b.}] From Schwarzschild metric equation for the stationary
    observer at $r = 3GM$ we have that
    \begin{align*}
        T &= \int d\tau = \int \sqrt{-ds^2}
        = \int \sqrt{\left(1 - \frac{2GM}{r}\right)dt^2} = \\
        &= \sqrt{1 - \frac{2}{3}} \int dt = 0.577 \Delta t
    \end{align*}
    Then the photon's speed $v$ as measured by the stationary observer is
    \begin{align*}
        v = \frac{rd\phi }{d\tau} = \frac{1}{0.577}\frac{rd\phi }{dt} = 1
    \end{align*}
    \item [\textbf{c.}] The velocities are not the same because the time is
    not measured by the same clocks. Velocity $V$ is computed using the
    time measuments of an observer at $r = \infty$ and velocity $v$ is
    computed using the time measurements of an stationary observer.
\end{itemize}
\end{proof}
\cleardoublepage
\begin{proof}{\textbf{P10.14}}
\begin{itemize}
    \item [\textbf{a.}] We know that in the equatorial plane
    $l = r^2 \dv{\phi}{\tau}$ then
    \begin{align*}
        \dv{\phi}{\tau} = \frac{l}{r^2}
    \end{align*}
    So for any finite $l$ as $r$ goes to infinity we have that
    $d\phi/d\tau \to 0$.\\    
    On the other hand, the Schwarzschild metric equation in the equatorial plane
    where $d\theta = 0$ gives us 
    \begin{align*}
        ds^2 = -\left(1 - \frac{2GM}{r}\right)dt^2
        + \left(1 - \frac{2GM}{r}\right)^{-1}dr^2 + r^2d\phi^2
    \end{align*}
    Also, we know that $d\tau^2 = -ds^2$ hence
    \begin{align*}
        d\tau^2 = \left(1 - \frac{2GM}{r}\right)dt^2
        - \left(1 - \frac{2GM}{r}\right)^{-1}dr^2 - r^2d\phi^2\\
        \left(1 - \frac{2GM}{r}\right)d\tau^2 = \left(1 - \frac{2GM}{r}\right)^2dt^2
        - dr^2 - \left(1 - \frac{2GM}{r}\right)r^2d\phi^2\\
        \left(1 - \frac{2GM}{r}\right)
        = \left(1 - \frac{2GM}{r}\right)^2\left(\dv{t}{\tau}\right)^2
        - \left(\dv{r}{\tau}\right)^2
        - \left(1 - \frac{2GM}{r}\right)r^2\left(\dv{\phi}{\tau}\right)^2\\
        \left(1 - \frac{2GM}{r}\right)
        = \left(1 - \frac{2GM}{r}\right)^2\left(\dv{t}{\tau}\right)^2
        - \left(\dv{r}{\tau}\right)^2
        - \left(1 - \frac{2GM}{r}\right)l\dv{\phi}{\tau}
    \end{align*}
    Now from the definition of $e$ and assuming that $dr/d\tau \approx 0$
    and $r \to \infty$ we get that
    \begin{align*}
        1 = e^2 - 0 - l\dv{\phi}{\tau}
    \end{align*}
    But also we saw that as $r \to \infty$ we have that $d\phi/d\tau \to 0$ hence
    \begin{align*}
        e = 1
    \end{align*}
    \item [\textbf{b.}] From the Schwarzschild metric for any $r$ and $e = 1$
    we know that 
    \begin{align*}
        \left(1 - \frac{2GM}{r}\right)\dv{t}{\tau} = 1
    \end{align*}
    Multiplying both sides by $l$ and using that $l = r^2d\phi/d\tau$ we get
    that
    \begin{align*}
        l = \left(1 - \frac{2GM}{r}\right)l\dv{t}{\tau}\\
        r^2\dv{\phi}{\tau} = \left(1 - \frac{2GM}{r}\right)l\dv{t}{\tau}\\
        \dv{\phi}{\tau} = \left(1 - \frac{2GM}{r}\right)\frac{l}{r^2}\dv{t}{\tau}
    \end{align*}
\cleardoublepage
    \item [\textbf{c.}] The relation $-1 = \bm{u} \cdot \bm{u}$ give us 
    \begin{align*}
        -\left(1 - \frac{2GM}{R}\right) (u^t)^2
        + \left(1 - \frac{2GM}{R}\right)^{-1} (u^r)^2 + R^2 (u^\theta)^2
        + R^2 \sin^2\theta(u^\phi)^2 = -1
    \end{align*}
    But assuming the path is in the equatorial plane and at the point of
    closest approach $u^r = 0$ and $u^\theta = 0$ we get that
    \begin{align*}
        -\left(1 - \frac{2GM}{R}\right) (u^t)^2 + R^2 (u^\phi)^2 &= -1\\
        -\left(1 - \frac{2GM}{R}\right) (u^t)^2
        + R^2 \left(\dv{\phi}{\tau}\right)^2 &= -1\\
        -\left(1 - \frac{2GM}{R}\right) (u^t)^2
        + R^2 \left(1 - \frac{2GM}{R}\right)^2\frac{l^2}{R^4}
        \bigg(\dv{t}{\tau}\bigg)^2 &= -1\\
        -\left(1 - \frac{2GM}{R}\right) (u^t)^2
        + \left(1 - \frac{2GM}{R}\right)^2\frac{l^2}{R^2} (u^t)^2 &= -1\\
        (u^t)^2\left(\left(1 - \frac{2GM}{R}\right)^2\frac{l^2}{R^2}
        - \left(1 - \frac{2GM}{R}\right) \right) &= -1
    \end{align*}
    Where we used the result of part b. Then
    \begin{align*}
        u^t &= \frac{1}{\sqrt{\left(1 - \frac{2GM}{R}\right)
        - \left(1 - \frac{2GM}{R}\right)^2\frac{l^2}{R^2}}}
    \end{align*}
    And therefore $p^t = mu^t$ is
    \begin{align*}
        p^t &= \frac{m}{\sqrt{\left(1 - \frac{2GM}{R}\right)
        - \left(1 - \frac{2GM}{R}\right)^2\frac{l^2}{R^2}}}
    \end{align*}
    \item [\textbf{d.}] For an observer at rest at $R$ we know that $u^t$ is
    the only non-zero component and from equation (9.20) we know that
    \begin{align*}
        u_{obs}^t = \left(1 - \frac{2GM}{R}\right)^{-1/2}
    \end{align*}
    Then $E_{obs}$ is 
    \begin{align*}
        E_{obs} &= -p^tu_{obs}^t \\
        &= m\sqrt{\frac{\left(1 - \frac{2GM}{R}\right)}{\left(1 - \frac{2GM}{R}\right)
        - \left(1 - \frac{2GM}{R}\right)^2\frac{l^2}{R^2}}}\\
        &= \frac{m}{\sqrt{1 - \left(1 - \frac{2GM}{R}\right)\frac{l^2}{R^2}}}
    \end{align*}
    \item [\textbf{e.}] Comparing the equation $E = m/\sqrt{1-v^2}$ to the result
    we got in part d we see that the speed $v$ as seen by and observer
    stationary at $R$ is
    \begin{align*}
        v &= \sqrt{\left(1 - \frac{2GM}{R}\right)\frac{l^2}{R^2}}
    \end{align*}
    \item [\textbf{f.}] Equation (10.8) for the closest approach becomes
    \begin{align*}
        \frac{1}{2}(1 -1) &= 0 - \frac{GM}{R} + \frac{l^2}{2R^2} - \frac{GMl^2}{R^3}\\
        0 &= \frac{2GMl^2}{R^2} - \frac{l^2}{R} + 2GM
    \end{align*}
    Where we used that $e = 1$ and $dr/d\tau = 0$ for the closest point. Hence
    taking $x = 1/R$ and solving the second degree equation we get that
    \begin{align*}
        x  &= \frac{l^2 \pm \sqrt{l^4 - 16(GM)^2l^2}}{4GMl^2}
        = \frac{1 \pm \sqrt{1 - 16(GM)^2/l^2}}{4GM}
    \end{align*}
    So changing to $R$ gives us
    \begin{align*}
        R = \frac{4GM}{1 \pm \sqrt{1 - 16(GM)^2/l^2}}
    \end{align*}
    There are two solutions because in the figure 10.1 the energy graph has two
    roots and we are essentially looking for them.
    Also, the smaller solution (the plus sign) is unstable so we take the bigger
    solution (the negative sign).
    
    If we consider a large-$l$ then we can approximate the square root by its
    binomial approximation i.e.
    \begin{align*}
        R = \frac{4GM}{1 - 1 + 8(GM)^2/l^2}
         = \frac{l^2}{2GM}
    \end{align*}
    Which gives us the Newtonian result.

    By analyzing figure 10.1 we see that the energy plot in the Schwarzschild
    case should have a smaller root compared to the Newtonian solution i.e.
    the Newtonian solution says $R$ should be farther this makes sense cause
    the Schwarzschild case takes into account the term $-GMl^2/R^3$.
\end{itemize}
\end{proof}
\cleardoublepage
\begin{proof}{\textbf{P10.15}}
\begin{itemize}
    \item [\textbf{a.}] If we start at rest at a very large raidus but with
    enough tangential velocity to give us some $l$ we saw in problem P10.8
    that we can spiral into an unstable circular orbit at $r = 4GM$ if
    $l = 4GM$ and since the orbit is unstable we will spiral back out again.
    
    For the portion of the trajectory where the orbit is approximately circular
    at $r = 4GM$ our clock measures a time $\tau$ and a clock at $r=\infty$
    measures a time $t$ hence by equation (10.5) and using that $e= 1$ since
    we start at rest we have that
    \begin{align*}
        \bigg(1 -\frac{2GM}{r}\bigg)\dv{t}{\tau} &= 1
    \end{align*}
    So integrating and replacing $r = 4GM$ we get that
    \begin{align*}
        \int d\tau &= \bigg(1 -\frac{2GM}{4GM}\bigg)\int dt\\
        \tau &= 0.5 t
    \end{align*}
    Then our clock will run slower by a factor of $0.5$ with respect to the
    clock at $r = \infty$.

    \item [\textbf{b.}] Suppose we want our clock to run 10 times more slowly 
    than the clock at large $r$ when we are at an approximately circular orbit
    then $\tau$ and $t$ are related by $t = 10\tau$.

    From problem P10.7(c) we know that 
    \begin{align*}
        e = \sqrt{1 - \frac{GM}{r}\left(1 - \frac{3GM}{r}\right)^{-1}
        \left[1 - \frac{4GM}{r}\right]}
    \end{align*}
    And joining this result to equation (10.5) we have that 
    \begin{align*}
        \left(1 - \frac{2GM}{r}\right)\dv{t}{\tau}
        &= \sqrt{1 - \frac{GM}{r}\left(1 - \frac{3GM}{r}\right)^{-1}
        \left[1 - \frac{4GM}{r}\right]}\\
        \left(1 - \frac{2GM}{r}\right)\int dt
        &= \sqrt{1 - \frac{GM}{r}\left(1 - \frac{3GM}{r}\right)^{-1}
        \left[1 - \frac{4GM}{r}\right]}\int d\tau \\
        t &= \left(1 - \frac{2GM}{r}\right)^{-1} 
        \sqrt{1 - \frac{GM}{r}\left(1 - \frac{3GM}{r}\right)^{-1}
        \left[1 - \frac{4GM}{r}\right]} \tau
    \end{align*}
    Then must happen that 
    \begin{align*}
        \left(1 - \frac{2GM}{r}\right)^{-1} 
        \sqrt{1 - \frac{GM}{r}\left(1 - \frac{3GM}{r}\right)^{-1}
        \left[1 - \frac{4GM}{r}\right]} &= 10
    \end{align*}
    Which by solving for $r$ we get that $r = 3.03GM$ hence $e = 3.41$.

    On the other hand, by the equation (10.32) we know that this implies an
    angular momentum of
    \begin{align*}
        l = \sqrt{\frac{9.18(GM)^3}{(3.03GM - 3GM)}}
        = \sqrt{\frac{9.18}{0.03}(GM)^2}
        = 17.49GM
    \end{align*}

    Finally, since at large $r$ the spacetime is essentially flat we can use
    that $e = (1 - v^2)^{-1/2}$ to determine $v$ hence
    \begin{align*} 
        v = \sqrt{1 - 1/e^2} = \sqrt{1 - 1/(3.41)^2} = 0.956
    \end{align*}
\end{itemize}
\end{proof}

\end{document}