\documentclass[11pt]{article}
\usepackage{amssymb}
\usepackage{amsthm}
\usepackage{enumitem}
\usepackage{amsmath}
\usepackage{bm}
\usepackage{adjustbox}
\usepackage{mathrsfs}
\usepackage{graphicx}
\usepackage{siunitx}
\usepackage[mathscr]{euscript}

\title{\textbf{Solved selected problems of General Relativity - Thomas A. Moore}}
\author{Franco Zacco}
\date{}

\addtolength{\topmargin}{-3cm}
\addtolength{\textheight}{3cm}

\newcommand{\hatr}{\bm{\hat{r}}}
\newcommand{\hatx}{\bm{\hat{x}}}
\newcommand{\haty}{\bm{\hat{y}}}
\newcommand{\hatz}{\bm{\hat{z}}}
\newcommand{\hatth}{\bm{\hat{\theta}}}
\newcommand{\hatphi}{\bm{\hat{\phi}}}
\newcommand{\hatrho}{\bm{\hat{\rho}}}
\theoremstyle{definition}
\newtheorem*{solution*}{Solution}
\renewcommand*{\proofname}{Solution}

\begin{document}
\maketitle
\thispagestyle{empty}

\section*{Chapter 2 - Review of Special Relativity}

\begin{proof}{\textbf{BOX 2.2}}
    We want to compute $g$ and $G$ in GR units, hence
    \begin{align*}
        g &= 9.81~\frac{m}{s^2}\left(\frac{1~s}{2.99792458 \times 10^8~m}\right)^2\\
          &= 1.091 \times 10^{-16}~m^{-1} = \frac{1}{9.17 \times 10^{15}~m}
    \end{align*}
    and since a light year is $9.4607 \times 10^{15}~m$ we could say that
    $g \approx 1~ly^{-1}$.
    
    Knowing that $N = kg~m/s^2$ for $G = 6.6743 \times 10^{-11}~Nm^2/kg^2$
    we have that
    \begin{align*}
        G &= 6.6743 \times 10^{-11} \frac{m^3}{s^2kg}
        \left(\frac{1~s}{2.99792458 \times 10^8~m}\right)^2\\
        &= 7.4261\times 10^{-28} \frac{m}{kg}
    \end{align*}
    and since 1 solar mass is equivalent to $1.988 \times 10^{30}~kg$
    we have that
    \begin{align*}
        G &= 7.4261\times 10^{-28} \frac{m}{kg}
        \left(\frac{1.988 \times 10^{30}~kg}{1~\text{solar mass}}\right)\\
        &= 1477~m/\text{solar mass}
    \end{align*}
\end{proof}
\cleardoublepage
\begin{proof}{\textbf{BOX 2.6}}
    Let $\Delta t > 0$ in the frame $S$ and $\Delta s^2 > 0$,
    the last inequality implies that
    $\Delta x^2 + \Delta y^2 + \Delta z^2 > \Delta t^2$ and since
    the events occur in the $+x$ axis we have that $\Delta x^2 > \Delta t^2$
    thus $\Delta x > \Delta t > 0$.
    
    Also, let us propose $\beta = \Delta t / \Delta x$ since we know that
    $0 < \Delta t < \Delta x$ then we have that 
    $0 < \beta = \Delta t/\Delta x < 1$.
    We want to prove next that it's possible to get $\Delta t' < 0$ for some
    $\beta$ such that $0 < \Delta t/ \Delta x < \beta < 1$. We see that
    \begin{align*}
        \Delta t &< \beta \Delta x\\
        \Delta t - \beta \Delta x &< 0\\
        \gamma\Delta t - \gamma\beta \Delta x &< 0
    \end{align*}
    But by the Lorentz transformations for a frame $S'$ moving at a speed
    $\beta$ with respect to $S$ we know that
    \begin{align*}
        \Delta t' = \gamma \Delta t - \gamma \beta \Delta x
    \end{align*}
    Therefore we have that $\Delta t' < 0$.

    If we now suppose $\Delta s^2 < 0$ then this implies that
    $\Delta x < \Delta t$ and for $\Delta t' < 0$ we know it must happen for
    $\beta$ that $\Delta t /\Delta x < \beta < 1$ but this is not possible
    since as we said $\Delta x < \Delta t$ implying that
    $1 < \Delta t/ \Delta x $. 
\end{proof}
\begin{proof}{\textbf{BOX 2.7}}
    We know that $d\tau = dt'$ since a clock carried by the particle should
    read the same time as an inertial clock in the same infinitesimal path
    then in terms of the spacetime interval we have that
    \begin{align*}
        d\tau = dt' &= \sqrt{-ds^2}\\
            &= \sqrt{dt^2 - dx^2 - dy^2 - dz^2}\\
            &= \sqrt{dt^2(1 - (dx/dt)^2 - (dy/dt)^2 - (dz/dt)^2)}\\
            &= dt\sqrt{1 - v^2}
    \end{align*}
    Where we used that $v^2 = (dx/dt)^2 + (dy/dt)^2 + (dz/dt)^2$
\end{proof}
\cleardoublepage
\begin{proof}{\textbf{P2.9}}
\begin{itemize}
    \item [(a)] We want to calculate the trip measured by clocks on A.
    We know that $v = 6/13$ since A travels
    at a constant speed and in a straight path from Alpha to Beta then
    using the proper time equation we have that
    \begin{align*}
        \Delta t_A &= \sqrt{1 - v^2}\int_0^{13} dt\\
        \Delta t_A &= 13 \sqrt{1 - (6/13)^2}\\ 
        \Delta t_A &= \sqrt{133}~\text{Tm} = 11.53~\text{Tm} 
    \end{align*}
    \item [(b)] We want to calculate now the trip measured by clocks on B.
    In this case, the semi-circular motion gives us a velocity
    $v = \omega r = (\pi/13)\cdot 3$ hence
    \begin{align*}
        \Delta t_B &= \sqrt{1 - v^2}\int_0^{13} dt\\
        \Delta t_B &= 13 \sqrt{1 - (3\pi/13)^2}\\
        \Delta t_B &= 8.95~\text{Tm} 
    \end{align*}
\end{itemize}
\end{proof}
\cleardoublepage
\begin{proof}{\textbf{P2.12}}
\begin{itemize}
    \item [(a)] Let us analyze a light ray that is going upwards in the $S'$
    frame then it does not have a velocity component in the $x'$ direction.
    Since $S'$ is travelling at a speed $\beta$, seen from frame $S$ this
    light ray will have a velocity component in the $x$ direction of magnitude
    $\beta$ and since the total velocity of the light ray must be $c = 1$
    then the velocity component in the $y$ direction must be
    $\sqrt{1 - \beta^2}$ so the angle with respect to the $x$ axis is given by
    \begin{align*}
        \sin(\theta) &= \frac{\sqrt{1 - \beta^2}}{1}\\
        \theta &= \sin^{-1}(\sqrt{1 - \beta^2})
    \end{align*}

    \item [(b)] Replacing the value of $\beta = 0.99$ in the above equation
    we get that
    \begin{align*}
        \theta &= \sin^{-1}(\sqrt{1 - (0.99)^2}) = 0.1415~\text{rad} = 8.10^\circ
    \end{align*}


\end{itemize}
\end{proof}
\end{document}


