\documentclass[11pt]{article}
\usepackage{amssymb}
\usepackage{amsthm}
\usepackage{enumitem}
\usepackage{physics,amsmath}
\usepackage{bm}
\usepackage{adjustbox}
\usepackage{mathrsfs}
\usepackage{graphicx}
\usepackage{siunitx}
\usepackage[mathscr]{euscript}


\title{\textbf{Solved selected problems of General Relativity - Thomas A. Moore}}
\author{Franco Zacco}
\date{}

\addtolength{\topmargin}{-3cm}
\addtolength{\textheight}{3cm}

\newcommand{\hatr}{\bm{\hat{r}}}
\newcommand{\hatn}{\bm{\hat{n}}}
\newcommand{\hatx}{\bm{\hat{x}}}
\newcommand{\haty}{\bm{\hat{y}}}
\newcommand{\hatz}{\bm{\hat{z}}}
\newcommand{\hatth}{\bm{\hat{\theta}}}
\newcommand{\hatphi}{\bm{\hat{\phi}}}
\newcommand{\hatrho}{\bm{\hat{\rho}}}
\newcommand{\er}{\bm{e}_r}
\newcommand{\etht}{\bm{e}_\theta}
\newcommand{\sci}[1]{\times 10^{#1}}

\theoremstyle{definition}
\newtheorem*{solution*}{Solution}
\renewcommand*{\proofname}{Solution}

\begin{document}
\maketitle
\thispagestyle{empty}

\section*{Chapter 17 - The Absolute Gradient}

\begin{proof}{\textbf{BOX 17.1} - Exercise 17.1.1.}\\
Let $d\bm{A} = d(A^{\mu}\bm{e}_\mu)$ then by the product rule we have that
\begin{align*}
    d(A^{\mu}\bm{e}_\mu)
    = \pdv{A^{\mu}}{x^\sigma}dx^{\sigma}\bm{e}_\mu
    + A^{\mu}\pdv{\bm{e}_\mu}{x^{\sigma}}dx^{\sigma}
\end{align*}
Also, by definition we know that
$\partial \bm{e}_\alpha/\partial x^{\mu} = \Gamma^\nu_{\mu\alpha}\bm{e}_\nu$
so
\begin{align*}
    d(A^{\mu}\bm{e}_\mu)
    &= \pdv{A^{\mu}}{x^\sigma}dx^{\sigma}\bm{e}_\mu
    + A^{\nu}\pdv{\bm{e}_\nu}{x^{\sigma}}dx^{\sigma}\\
    &= \pdv{A^{\mu}}{x^\sigma}dx^{\sigma}\bm{e}_\mu
    + A^{\nu}\Gamma^{\mu}_{\sigma\nu}\bm{e}_{\mu} dx^{\sigma}\\
    &= \bigg[\pdv{A^{\mu}}{x^\sigma} + A^{\nu}\Gamma^{\mu}_{\sigma\nu}\bigg]
    \bm{e}_{\mu} dx^{\sigma}
\end{align*}
Where in the first step we renamed in the second term $\mu$ to $\nu$.
\end{proof}
\begin{proof}{\textbf{BOX 17.2} - Exercise 17.2.1.}\\
From equation (17.17) we know that
\begin{align*}
    \nabla_\alpha(A^\mu B_\mu)
    = \bigg[\pdv{A^\mu}{x^{\alpha}} + \Gamma^\mu_{\alpha\nu}A^\nu\bigg]B_\mu
    + A^\mu(\nabla_\alpha B_\mu)
\end{align*}
And from equation (17.18) we know that
\begin{align*}
    \nabla_\alpha(A^\mu B_\mu) = \pdv{A^\mu}{x^\alpha}B_\mu + A^\mu\pdv{B_\mu}{x^\alpha}
\end{align*}
Then
\begin{align*}
    \pdv{A^\mu}{x^{\alpha}}B_\mu + \Gamma^\mu_{\alpha\nu}A^\nu B_\mu
    + A^\mu(\nabla_\alpha B_\mu)
    &= \pdv{A^\mu}{x^\alpha}B_\mu + A^\mu\pdv{B_\mu}{x^\alpha}\\
    \Gamma^\mu_{\alpha\nu}A^\nu B_\mu + A^\mu(\nabla_\alpha B_\mu)
    &= A^\mu\pdv{B_\mu}{x^\alpha}\\
    - A^\mu\pdv{B_\mu}{x^\alpha} + \Gamma^\sigma_{\alpha\mu}A^\mu B_\sigma
    + A^\mu(\nabla_\alpha B_\mu) &= 0\\
    A^\mu\bigg[-\pdv{B_\mu}{x^\alpha} + \Gamma^\sigma_{\alpha\mu}B_\sigma
    + \nabla_\alpha B_\mu \bigg] &= 0
\end{align*}
Where in the third step we renamed the $\nu$ index to $\mu$ and the $\mu$
index to $\sigma$.
\end{proof}

\cleardoublepage
\begin{proof}{\textbf{BOX 17.4} - Exercise 17.4.1.}\\
Equations (17.22), (17.23) and (17.24) state that
\begin{align*}
    \Gamma_{\mu\alpha}^\nu g_{\nu\rho} + \Gamma_{\rho\alpha}^\nu g_{\nu\mu}
    &= \partial_{\alpha}g_{\mu\rho}\\
    %
    \Gamma_{\rho\mu}^\nu g_{\nu\alpha} + \Gamma_{\alpha\mu}^\nu g_{\nu\rho}
    &= \partial_{\mu}g_{\rho\alpha}\\
    %
    \Gamma_{\alpha\rho}^\nu g_{\nu\mu} + \Gamma_{\mu\rho}^\nu g_{\nu\alpha}
    &= \partial_{\rho}g_{\alpha\mu}
\end{align*}
Then adding (17.22) to (17.23) and subtracting (17.24) we get that
\begin{align*}
    \Gamma_{\mu\alpha}^\nu g_{\nu\rho} + \Gamma_{\rho\alpha}^\nu g_{\nu\mu}
    + \Gamma_{\rho\mu}^\nu g_{\nu\alpha} + \Gamma_{\alpha\mu}^\nu g_{\nu\rho}
    - \Gamma_{\alpha\rho}^\nu g_{\nu\mu} - \Gamma_{\mu\rho}^\nu g_{\nu\alpha}
    &= \partial_{\alpha}g_{\mu\rho} + \partial_{\mu}g_{\rho\alpha}
    - \partial_{\rho}g_{\alpha\mu}\\
    %
    g_{\nu\rho}(\Gamma_{\mu\alpha}^\nu + \Gamma_{\alpha\mu}^\nu)
    + g_{\nu\mu}(\Gamma_{\rho\alpha}^\nu - \Gamma_{\alpha\rho}^\nu)
    + g_{\nu\alpha}(\Gamma_{\rho\mu}^\nu - \Gamma_{\mu\rho}^\nu)
    &= \partial_{\alpha}g_{\mu\rho} + \partial_{\mu}g_{\rho\alpha}
    - \partial_{\rho}g_{\alpha\mu}\\
    %
    g_{\nu\rho}(\Gamma_{\mu\alpha}^\nu + \Gamma_{\mu\alpha}^\nu)
    + g_{\nu\mu}(\Gamma_{\rho\alpha}^\nu - \Gamma_{\rho\alpha}^\nu)
    + g_{\nu\alpha}(\Gamma_{\rho\mu}^\nu - \Gamma_{\rho\mu}^\nu)
    &= \partial_{\alpha}g_{\mu\rho} + \partial_{\mu}g_{\rho\alpha}
    - \partial_{\rho}g_{\alpha\mu}
\end{align*}
Where we used the symmetry in the lower indices of the Christoffel symbols,
then
\begin{align*}
    2g_{\nu\rho}\Gamma_{\mu\alpha}^\nu
    &= \partial_{\alpha}g_{\mu\rho} + \partial_{\mu}g_{\rho\alpha}
    - \partial_{\rho}g_{\alpha\mu}
\end{align*}
So multiplying this equation by $\frac{1}{2}g^{\sigma\rho}$ and using that
$g^{\sigma\rho}g_{\nu\rho} = \delta^\sigma_\nu$ we get that
\begin{align*}
    g^{\sigma\rho}g_{\nu\rho}\Gamma_{\mu\alpha}^\nu
    &= \frac{1}{2}g^{\sigma\rho}[
    \partial_{\alpha}g_{\mu\rho} + \partial_{\mu}g_{\rho\alpha}
    - \partial_{\rho}g_{\alpha\mu}
    ]\\
    \delta^\sigma_\nu\Gamma_{\mu\alpha}^\nu
    &= \frac{1}{2}g^{\sigma\rho}[
    \partial_{\alpha}g_{\mu\rho} + \partial_{\mu}g_{\rho\alpha}
    - \partial_{\rho}g_{\alpha\mu}
    ]\\
    \Gamma_{\mu\alpha}^\sigma
    &= \frac{1}{2}g^{\sigma\rho}[
    \partial_{\alpha}g_{\mu\rho} + \partial_{\mu}g_{\rho\alpha}
    - \partial_{\rho}g_{\alpha\mu}
    ]
\end{align*}
\end{proof}

\cleardoublepage
\begin{proof}{\textbf{BOX 17.5} - Exercise 17.5.1.}\\
Equation (17.27) states that
\begin{align*}
    \frac{1}{2}\partial_\sigma g_{\mu\nu} \dv{x^\sigma}{\tau}\dv{x^\nu}{\tau}
    + \frac{1}{2}\partial_\sigma g_{\mu\nu} \dv{x^\sigma}{\tau}\dv{x^\nu}{\tau}
    + g_{\mu\nu}\dv[2]{x^\nu}{\tau}
    - \frac{1}{2}\partial_\mu g_{\alpha\beta} \dv{x^\alpha}{\tau}\dv{x^\beta}{\tau}
    = 0
\end{align*}
Renaming $\sigma \to \alpha$, $\nu\to\beta$ in the first term and
$\sigma \to \beta, \nu \to \alpha$ in the second term we get that
\begin{align*}
    \frac{1}{2}\partial_\alpha g_{\mu\beta} \dv{x^\alpha}{\tau}\dv{x^\beta}{\tau}
    + \frac{1}{2}\partial_\beta g_{\mu\alpha} \dv{x^\beta}{\tau}\dv{x^\alpha}{\tau}
    + g_{\mu\nu}\dv[2]{x^\nu}{\tau}
    - \frac{1}{2}\partial_\mu g_{\alpha\beta} \dv{x^\alpha}{\tau}\dv{x^\beta}{\tau}
    = 0
\end{align*}
So, multiplying the equation by $g^{\sigma\mu}$ and using the fact that
$g^{\sigma\mu}g_{\mu\nu} = \delta^\sigma_\nu$ we see that
\begin{align*}
    \frac{1}{2}g^{\sigma\mu}\bigg[
        \partial_\alpha g_{\mu\beta} \dv{x^\alpha}{\tau}\dv{x^\beta}{\tau}
        + \partial_\beta g_{\mu\alpha} \dv{x^\beta}{\tau}\dv{x^\alpha}{\tau}
        - \partial_\mu g_{\alpha\beta} \dv{x^\alpha}{\tau}\dv{x^\beta}{\tau}
    \bigg]
    + \delta^\sigma_\nu\dv[2]{x^\nu}{\tau}
    &= 0\\
    \frac{1}{2}g^{\sigma\mu}\bigg[
        \partial_\alpha g_{\mu\beta} + \partial_\beta g_{\mu\alpha}
        - \partial_\mu g_{\alpha\beta}
    \bigg]\dv{x^\alpha}{\tau}\dv{x^\beta}{\tau}
    + \dv[2]{x^\sigma}{\tau}
    &= 0
\end{align*}
Finally, using equation (17.10) we have that
\begin{align*}
    \Gamma_{\alpha\beta}^\sigma\dv{x^\alpha}{\tau}\dv{x^\beta}{\tau}
    + \dv[2]{x^\sigma}{\tau}
    &= 0
\end{align*}
Which renaming $\sigma \to \mu$ gives us equation (17.12).
\end{proof}

\cleardoublepage
\begin{proof}{\textbf{BOX 17.6} - Exercise 17.6.1.}\\
Let $\mu = \theta$ then the geodesic equation becomes
\begin{align*}
    \partial_\sigma g_{\theta\nu} \dv{x^\sigma}{\tau}\dv{x^\nu}{\tau}
    + g_{\theta\nu}\dv[2]{x^\nu}{\tau}
    - \frac{1}{2}\partial_\theta g_{\alpha\beta} \dv{x^\alpha}{\tau}\dv{x^\beta}{\tau}
    = 0
\end{align*}
For the first and second term, since the metric is diagonal the only non-zero
term is $g_{\theta\theta}$ and this term only depends on $r$ so the only
derivative that is non-zero is when $\sigma = r$.
\\
For the last term, the only component that depends on $\theta$ is the
$g_{\phi\phi}$ component, then the equation reduces to
\begin{align*}
    \partial_r g_{\theta\theta} \dv{r}{\tau}\dv{\theta}{\tau}
    + g_{\theta\theta}\dv[2]{\theta}{\tau}
    - \frac{1}{2}\partial_\theta g_{\phi\phi} \dv{\phi}{\tau}\dv{\phi}{\tau}
    &= 0\\
    \partial_r (r^2) \dv{r}{\tau}\dv{\theta}{\tau}
    + r^2 \dv[2]{\theta}{\tau}
    - \frac{1}{2}\partial_\theta (r^2\sin^2\theta)\dv{\phi}{\tau}\dv{\phi}{\tau}
    &= 0\\
    2r \dv{r}{\tau}\dv{\theta}{\tau} + r^2 \dv[2]{\theta}{\tau}
    - r^2\sin\theta\cos\theta \dv{\phi}{\tau}\dv{\phi}{\tau}
    &= 0\\
    \frac{2}{r} \dv{r}{\tau}\dv{\theta}{\tau} + \dv[2]{\theta}{\tau}
    - \sin\theta\cos\theta \dv{\phi}{\tau}\dv{\phi}{\tau}
    &= 0
\end{align*}
Therefore, from the 16 Christoffel symbols $\Gamma^\theta_{\mu\nu}$ that have
$\theta$ as a superscript the only non-zero terms are
\begin{align*}
    \Gamma^\theta_{r\theta} = \Gamma^\theta_{\theta r} = \frac{1}{r}
    \quad\text{ and }\quad
    \Gamma^\theta_{\phi\phi} = -\sin\theta\cos\theta
\end{align*}

\end{proof}

\cleardoublepage
\begin{proof}{\textbf{BOX 17.7} - Exercise 17.7.1.}\\
Let us list all 20 valid combinations of the three indices as follows
\begin{align*}
    333 && 233 && 322 && 311 && 300 && 012\\ 
    222 && 133 && 122 && 211 && 200 && 123\\ 
    111 && 033 && 022 && 011 && 100 && 230\\ 
    000 &&     &&     &&     &&     && 301
\end{align*}
\end{proof}
\begin{proof}{\textbf{BOX 17.7} - Exercise 17.7.2.}\\
From equation (17.35) we know that
\begin{align*}
    \bm{g}' = [\bm{a} + \bm{b}\Delta\bm{x}' + \bm{c}(\Delta\bm{x}')^2 + ...]^2
    [\bm{g}_P + \partial\bm{g}_P\Delta\bm{x}'
    + \frac{1}{2}\partial^2\bm{g}_P(\Delta\bm{x}')^2 + ...]
\end{align*}
Then expanding this expression we get that
\begin{align*}
    &\bm{g}' =
    [\bm{a}^2 + \bm{a}\bm{b}\Delta\bm{x}' + \bm{a}\bm{c}(\Delta\bm{x}')^2
    + \bm{b}\bm{a}\Delta\bm{x}' + \bm{b}^2(\Delta\bm{x}')^2
    + \bm{b}\bm{c}(\Delta\bm{x}')^3 + \bm{c}\bm{a}(\Delta\bm{x}')^2\\
    &\qquad + \bm{c}\bm{b}(\Delta\bm{x}')^3 + \bm{c}^2(\Delta\bm{x}')^4
    + ...][\bm{g}_P + \partial\bm{g}_P\Delta\bm{x}'
    + \frac{1}{2}\partial^2\bm{g}_P(\Delta\bm{x}')^2 + ...]\\
    &\quad =
    [\bm{a}^2 + [\bm{a}\bm{b} + \bm{b}\bm{a}]\Delta\bm{x}'
    + [\bm{a}\bm{c} + \bm{b}^2 + \bm{c}\bm{a}](\Delta\bm{x}')^2 + ...]
    [\bm{g}_P + \partial\bm{g}_P\Delta\bm{x}'
    + \frac{1}{2}\partial^2\bm{g}_P(\Delta\bm{x}')^2 + ...]\\
    &\quad = \bm{a}^2\bm{g}_P + \bm{a}^2\partial\bm{g}_P\Delta\bm{x}'
    + \bm{a}^2\frac{1}{2}\partial^2\bm{g}_P(\Delta\bm{x}')^2
    + [\bm{a}\bm{b} + \bm{b}\bm{a}]\bm{g}_P\Delta\bm{x}'
    + [\bm{a}\bm{b} + \bm{b}\bm{a}]\partial\bm{g}_P(\Delta\bm{x}')^2\\
    &\qquad + \bm{g}_P[\bm{a}\bm{c} + \bm{b}^2 + \bm{c}\bm{a}](\Delta\bm{x}')^2
    + ...\\
    &\quad = \bm{a}^2\bm{g}_P
    + [\bm{a}^2\partial\bm{g}_P + [\bm{a}\bm{b} + \bm{b}\bm{a}]\bm{g}_P]\Delta\bm{x}'\\
    &\qquad+ [\bm{a}^2\frac{1}{2}\partial^2\bm{g}_P
    + [\bm{a}\bm{b} + \bm{b}\bm{a}]\partial\bm{g}_P
    + \bm{g}_P[\bm{a}\bm{c} + \bm{b}^2 + \bm{c}\bm{a}]](\Delta\bm{x}')^2
    + ...
\end{align*}
Where we removed the higher order terms. Therefore we get that
\begin{align*}
    \bm{g}' &= \bm{a}^2\bm{g}_P
    + [\bm{a}^2\partial\bm{g}_P + \bm{a}\bm{b}\bm{g}_P
    + \bm{b}\bm{a}\bm{g}_P]\Delta\bm{x}'\\
    &\quad+ [\frac{1}{2}\bm{a}^2\partial^2\bm{g}_P
    + \bm{a}\bm{b}\partial\bm{g}_P + \bm{b}\bm{a}\partial\bm{g}_P
    + \bm{a}\bm{c}\bm{g}_P + \bm{b}^2\bm{g}_P + \bm{c}\bm{a}\bm{g}_P
    ](\Delta\bm{x}')^2
    + ...
\end{align*}
\end{proof}

\cleardoublepage
\begin{proof}{\textbf{P17.1}}\\
We know that in polar-coordinates
\begin{align*}
    ds^2 = dr^2 + r^2d\theta^2
\end{align*}
So the geodesic equation for $\mu = r$ is
\begin{align*}
    \partial_\sigma g_{r\nu} \dv{x^\sigma}{\tau}\dv{x^\nu}{\tau}
    + g_{r\nu}\dv[2]{x^\nu}{\tau}
    - \frac{1}{2}\partial_r g_{\alpha\beta} \dv{x^\alpha}{\tau}\dv{x^\beta}{\tau}
    = 0
\end{align*}
For the first and second term, since the metric is diagonal the only non-zero
term is $g_{rr}$ and it's constant so every derivative is 0.
\\
For the last term, the only component that depends on $r$ is the
$g_{\theta\theta}$ component, then the equation reduces to
\begin{align*}
    g_{rr}\dv[2]{r}{\tau}
    - \frac{1}{2}\partial_r g_{\theta\theta} \dv{\theta}{\tau}\dv{\theta}{\tau}
    &= 0\\
    \dv[2]{r}{\tau} - r \dv{\theta}{\tau}\dv{\theta}{\tau}
    &= 0
\end{align*}
Comparing this to the 4 Christoffel symbols $\Gamma^r_{\mu\nu}$ that have $r$
as a superscript the only-non-zero term is
\begin{align*}
    \Gamma^r_{\theta\theta} = -r
\end{align*}
Now, if we let $\mu = \theta$ the geodesic equation becomes
\begin{align*}
    \partial_\sigma g_{\theta\nu} \dv{x^\sigma}{\tau}\dv{x^\nu}{\tau}
    + g_{\theta\nu}\dv[2]{x^\nu}{\tau}
    - \frac{1}{2}\partial_\theta g_{\alpha\beta} \dv{x^\alpha}{\tau}\dv{x^\beta}{\tau}
    = 0
\end{align*}
For the first and second term, since the metric is diagonal the only non-zero
term is $g_{\theta\theta}$ and it depends only on $r$ so the $r$-derivative 
is the only non-zero derivative.
\\
For the last term, no component of the metric depends on $\theta$ so all
derivatives with respect to $\theta$ are zero. Then the equation becomes
\begin{align*}
    \partial_r g_{\theta\theta} \dv{r}{\tau}\dv{\theta}{\tau}
    + g_{\theta\theta}\dv[2]{\theta}{\tau}
    &= 0\\
    2r \dv{r}{\tau}\dv{\theta}{\tau}
    + r^2\dv[2]{\theta}{\tau}
    &= 0\\
    \frac{2}{r}\dv{r}{\tau}\dv{\theta}{\tau}
    + \dv[2]{\theta}{\tau}
    &= 0
\end{align*}
Comparing this to the 4 Christoffel symbols $\Gamma^\theta_{\mu\nu}$ that have
$\theta$ as a superscript the only-non-zero terms are
\begin{align*}
    \Gamma^\theta_{r\theta} = \Gamma^\theta_{\theta r} = \frac{1}{r} 
\end{align*}
Therefore, from the 8 Christoffel symbols for polar coordinates only 3 are
non-zero.
Finally, we can check that the result we got for $\Gamma^r_{\theta\theta}$
matches to the result given by equation (17.10) by computing it as follows
\begin{align*}
    \Gamma^r_{\theta\theta}
    = \frac{1}{2} g^{r\sigma}[
    \partial_\theta g_{\theta\sigma} + \partial_\theta g_{\sigma\theta}
    - \partial_\sigma g_{\theta\theta}
    ]
\end{align*}
Given that no metric component depends on $\theta$ we can drop the first two
terms, and since the metric inverse is also diagonal the only non-zero term is
$g^{rr}$, then we get that
\begin{align*}
    \Gamma^r_{\theta\theta}
    &= -\frac{1}{2} g^{rr} \partial_r g_{\theta\theta}
    = -\frac{1}{2} \frac{1}{g_{rr}} (2r)
    = -\frac{1}{2} \cdot 1 \cdot (2r)
    = -r
\end{align*}
Since the metric is diagonal, we used that $g^{\mu\nu} = 1/g_{\mu\nu}$ and we
see that the result matches the previous one.
\end{proof}

\cleardoublepage
\begin{proof}{\textbf{P17.2}}\\
We computed the Christoffel symbols for when $t$ and $\theta$ are superscripts
in BOX 17.6, so we have to compute Christoffel symbols for when $r$ and $\phi$
are superscripts.
\\
Let us start considering the metric equation for $\mu = r$.
\begin{align*}
    \partial_\sigma g_{r\nu} \dv{x^\sigma}{\tau}\dv{x^\nu}{\tau}
    + g_{r\nu}\dv[2]{x^\nu}{\tau}
    - \frac{1}{2}\partial_r g_{\alpha\beta} \dv{x^\alpha}{\tau}\dv{x^\beta}{\tau}
    = 0
\end{align*}
For the first and second term, since the metric is diagonal the only non-zero
term is $g_{rr}$ and this term only depends on $r$ so the only
derivative that is non-zero is when $\sigma = r$.
\\
Then the equation reduces to
\begin{align*}
    &\partial_r g_{rr} \dv{r}{\tau}\dv{r}{\tau}
    + g_{rr}\dv[2]{r}{\tau}\\
    &\quad- \frac{1}{2}\bigg[
    \partial_r g_{tt} \dv{t}{\tau}\dv{t}{\tau}
    + \partial_r g_{rr} \dv{r}{\tau}\dv{r}{\tau}
    + \partial_r g_{\theta\theta} \dv{\theta}{\tau}\dv{\theta}{\tau}
    + \partial_r g_{\phi\phi} \dv{\phi}{\tau}\dv{\phi}{\tau}
    \bigg]
    = 0\\[10pt]
    &-\frac{2GM}{r^2}\bigg(1 - \frac{2GM}{r}\bigg)^{-2}\dv{r}{\tau}\dv{r}{\tau}
    + \bigg(1 - \frac{2GM}{r}\bigg)^{-1}\dv[2]{r}{\tau}\\
    &\quad- \frac{1}{2}\bigg[
    -\frac{2GM}{r^2} \dv{t}{\tau}\dv{t}{\tau}
    -\frac{2GM}{r^2}\bigg(1 - \frac{2GM}{r}\bigg)^{-2}\dv{r}{\tau}\dv{r}{\tau}
    + 2r \dv{\theta}{\tau}\dv{\theta}{\tau}
    + 2r\sin^2\theta \dv{\phi}{\tau}\dv{\phi}{\tau}
    \bigg]
    = 0\\[10pt]
    &-\frac{2GM}{r^2}\bigg(1 - \frac{2GM}{r}\bigg)^{-1}\dv{r}{\tau}\dv{r}{\tau}
    + \dv[2]{r}{\tau}\\
    &\quad- \frac{1}{2}\bigg[
    -\frac{2GM}{r^2}\bigg(1 - \frac{2GM}{r}\bigg) \dv{t}{\tau}\dv{t}{\tau}
    -\frac{2GM}{r^2}\bigg(1 - \frac{2GM}{r}\bigg)^{-1}\dv{r}{\tau}\dv{r}{\tau}\\
    &\quad+ 2r\bigg(1 - \frac{2GM}{r}\bigg) \dv{\theta}{\tau}\dv{\theta}{\tau}
    + 2r\sin^2\theta\bigg(1 - \frac{2GM}{r}\bigg)\dv{\phi}{\tau}\dv{\phi}{\tau}
    \bigg]
    = 0\\[10pt]
    &-\frac{GM}{r^2}\bigg(1 - \frac{2GM}{r}\bigg)^{-1}\dv{r}{\tau}\dv{r}{\tau}
    + \dv[2]{r}{\tau}
    + \frac{GM}{r^2}\bigg(1 - \frac{2GM}{r}\bigg) \dv{t}{\tau}\dv{t}{\tau}\\
    &\quad- r\bigg(1 - \frac{2GM}{r}\bigg) \dv{\theta}{\tau}\dv{\theta}{\tau}
    - r\sin^2\theta\bigg(1 - \frac{2GM}{r}\bigg)\dv{\phi}{\tau}\dv{\phi}{\tau}
    = 0
\end{align*}
So from the 16 Christoffel symbols $\Gamma^r_{\mu\nu}$ that have $r$ as a
superscript the only non-zero terms are
\begin{align*}
    \Gamma^r_{tt} &= \frac{GM}{r^2}\bigg(1 - \frac{2GM}{r}\bigg)
    &\qquad
    \Gamma^r_{rr} &= -\frac{GM}{r^2}\bigg(1 - \frac{2GM}{r}\bigg)^{-1}\\
    \Gamma^r_{\theta\theta} &= -r\bigg(1 - \frac{2GM}{r}\bigg)
    &\qquad
    \Gamma^r_{\phi\phi} &= -r\sin^2\theta\bigg(1 - \frac{2GM}{r}\bigg)
\end{align*}
Now, let us consider the metric equation for $\mu = \phi$.
\begin{align*}
    \partial_\sigma g_{\phi\nu} \dv{x^\sigma}{\tau}\dv{x^\nu}{\tau}
    + g_{\phi\nu}\dv[2]{x^\nu}{\tau}
    - \frac{1}{2}\partial_\phi g_{\alpha\beta} \dv{x^\alpha}{\tau}\dv{x^\beta}{\tau}
    = 0
\end{align*}
For the first and second term, since the metric is diagonal the only non-zero
term is $g_{\phi\phi}$ and this term depends only on $r$ and $\theta$
so the only derivatives that are non-zero are when $\sigma = r, \theta$.
\\
For the last term, no component of the metric depends on $\phi$ so all
derivatives with respect to $\phi$ are zero.
Then the equation becomes
\begin{align*}
    \partial_r g_{\phi\phi} \dv{r}{\tau}\dv{\phi}{\tau}
    + \partial_\theta g_{\phi\phi} \dv{\theta}{\tau}\dv{\phi}{\tau}
    + g_{\phi\phi}\dv[2]{\phi}{\tau}
    &= 0\\
    2r\sin^2\theta \dv{r}{\tau}\dv{\phi}{\tau}
    + 2r^2\sin\theta\cos\theta \dv{\theta}{\tau}\dv{\phi}{\tau}
    + r^2\sin^2\theta\dv[2]{\phi}{\tau}
    &= 0\\
    \frac{2}{r}\dv{r}{\tau}\dv{\phi}{\tau}
    + 2\frac{\cos\theta}{\sin\theta} \dv{\theta}{\tau}\dv{\phi}{\tau}
    + \dv[2]{\phi}{\tau}
    &= 0\\
    \frac{2}{r}\dv{r}{\tau}\dv{\phi}{\tau}
    + 2\cot\theta \dv{\theta}{\tau}\dv{\phi}{\tau}
    + \dv[2]{\phi}{\tau}
    &= 0
\end{align*}
Therefore from the 16 Christoffel symbols $\Gamma^\phi_{\mu\nu}$ that have
$\phi$ as a superscript the only non-zero terms are
\begin{align*}
    \Gamma^\phi_{r\phi} = \Gamma^\phi_{\phi r} &= \frac{1}{r}
    \qquad\qquad    
    \Gamma^\phi_{\theta\phi} = \Gamma^\phi_{\phi\theta} = \cot\theta
\end{align*}
Finally, we can check that the result we got for $\Gamma^\phi_{r\phi}$
matches to the result given by equation (17.10) by computing it as follows
\begin{align*}
    \Gamma^\phi_{r\phi}
    = \frac{1}{2} g^{\phi\sigma}[
    \partial_r g_{\phi\sigma} + \partial_\phi g_{\sigma r}
    - \partial_\sigma g_{r\phi}]
\end{align*}
Given that no metric component depends on $\phi$ we can drop the second term,
the third term is zero as well since the metric is diagonal.
Also, since the metric is diagonal the only non-zero terms are when
$\sigma = \phi$, then the equation becomes
\begin{align*}
    \Gamma^\phi_{r\phi}
    = \frac{1}{2} g^{\phi\phi}[\partial_r g_{\phi\phi}]
    = \frac{1}{2} \frac{1}{g_{\phi\phi}}[2r\sin^2\theta]
    = \frac{1}{2} \frac{1}{r^2\sin^2\theta}[2r\sin^2\theta]
    = \frac{1}{r}
\end{align*}
Where we used that $g^{\mu\nu} = 1/g_{\mu\nu}$ since the metric is diagonal.
\end{proof}

\cleardoublepage
\begin{proof}{\textbf{P17.4}}\\
Let us take a LIF centered on a given event, then the absolute gradient reduces
to the ordinary gradient and hence we get that
\begin{align*}
    \nabla_\beta(B^\mu_\nu A^\alpha)
    &= \partial_\beta(B^\mu_\nu A^\alpha)\\
    &= (\partial_\beta B^\mu_\nu) A^\alpha + (\partial_\beta A^\alpha)B^\mu_\nu\\
    &= (\nabla_\beta B^\mu_\nu) A^\alpha + (\nabla_\beta A^\alpha)B^\mu_\nu
\end{align*}
Where we used the product rule of the ordinary gradient.
Since the last equation is a tensor equation, if it holds in any coordinate
system, it must hold in all. Therefore the absolute gradient obeys the product
rule.
\end{proof}

\cleardoublepage
\begin{proof}{\textbf{P17.6}}\\
The correct way of computing $\bm{a}$ is by using equation (17.12) where we
compute $d\bm{v}/dt$ taking into account the change in the basis vectors
\begin{align*}
    a^\mu = \dv{\bm{v}}{t} = \dv{v^\mu}{t} + \Gamma^\mu_{\alpha\beta}v^\alpha v^\beta
\end{align*}
Let us consider first the case of cartesian coordinates.
\\
We see that the Christoffel symbols can be calculated using the equation
\begin{align*}
    \Gamma^\alpha_{\mu\nu} = \frac{1}{2} g^{\alpha\sigma}[
    \partial_\mu g_{\nu\sigma} + \partial_\nu g_{\sigma \mu}
    - \partial_\sigma g_{\mu\nu}]
\end{align*}
Since all components of the metric are constant (either 0 or 1) then all
derivatives are zero and hence the Christoffel components are all zero, hence
\begin{align*}
    a^\mu = \dv{v^\mu}{t} = 0
\end{align*}
Where we used that $v^x = v$ and $v^y = 0$ i.e. the object is moving with
constant velocity in cartesian coordinates.
\\
So, as expected, the components of $\bm{a}$ are zero in cartesian coordinates.
\\
Now, let us compute the acceleration in the sinusoidal coordinates $u, w$
where $u = x$ and $w = y - A\sin(bx)$ then from problem P5.5 we know that
metric in matrix form is
\begin{align*}
    g_{\mu\nu} &= \begin{bmatrix}
        1 + (Ab)^2\cos^2(bu) & Ab\cos(bu)\\
        Ab\cos(bu) & 1
    \end{bmatrix}
\end{align*}
We see that $\det(g_{\mu\nu}) = 1$ so the inverse metric is given by
\begin{align*}
    g^{\mu\nu} &= \begin{bmatrix}
        1 & -Ab\cos(bu)\\
        -Ab\cos(bu) & 1 + (Ab)^2\cos^2(bu)
    \end{bmatrix}
\end{align*}
For the case $\alpha = u$ we can compute the Christoffel symbols as follows
\begin{align*}
    \Gamma^u_{uu}
    &= \frac{1}{2}g^{uu}[\partial_ug_{uu} + \partial_ug_{uu} - \partial_ug_{uu}]
    + \frac{1}{2}g^{uw}[\partial_ug_{uw} + \partial_ug_{wu} - \partial_wg_{uu}]\\
    &= \frac{1}{2}g^{uu}\partial_ug_{uu} + g^{uw}\partial_ug_{uw}\\
    &= -A^2b^3\sin(bu)\cos(bu) + A^2b^3\cos(bu)\sin(bu)\\
    &= 0
\end{align*}
\begin{align*}
    \Gamma^u_{uw}
    &= \frac{1}{2}g^{uu}[\partial_ug_{wu} + \partial_wg_{uu} - \partial_ug_{uw}]
    + \frac{1}{2}g^{uw}[\partial_ug_{ww} + \partial_wg_{wu} - \partial_wg_{uw}]\\
    &= 0
\end{align*}
\begin{align*}
    \Gamma^u_{wu}
    &= \frac{1}{2}g^{uu}[\partial_wg_{uu} + \partial_ug_{uw} - \partial_ug_{wu}]
    + \frac{1}{2}g^{uw}[\partial_wg_{uw} + \partial_ug_{ww} - \partial_wg_{wu}]\\
    &= 0
\end{align*}
\begin{align*}
    \Gamma^u_{ww}
    &= \frac{1}{2}g^{uu}[\partial_wg_{wu} + \partial_wg_{uw} - \partial_ug_{ww}]
    + \frac{1}{2}g^{uw}[\partial_wg_{ww} + \partial_wg_{ww} - \partial_wg_{ww}]\\
    &= 0
\end{align*}
For the case $\alpha = w$ we can compute the Christoffel symbols as follows
\begin{align*}
    \Gamma^w_{uu}
    &= \frac{1}{2}g^{wu}[\partial_ug_{uu} + \partial_ug_{uu} - \partial_ug_{uu}]
    + \frac{1}{2}g^{ww}[\partial_ug_{uw} + \partial_ug_{wu} - \partial_wg_{uu}]\\
    &= \frac{1}{2}g^{wu}\partial_ug_{uu} + g^{ww}\partial_ug_{uw}\\
    &= Ab\cos(bu) A^2b^3\cos(bu)\sin(bu) - (1 + (Ab)^2\cos^2(bu))Ab^2\sin(bu)\\
    &= A^3b^4\cos^2(bu)\sin(bu) - Ab^2\sin(bu) - A^3b^4\cos^2(bu)\sin(bu)\\
    &= - Ab^2\sin(bu)
\end{align*}
\begin{align*}
    \Gamma^w_{uw}
    &= \frac{1}{2}g^{wu}[\partial_ug_{wu} + \partial_wg_{u} - \partial_ug_{uw}]
    + \frac{1}{2}g^{ww}[\partial_ug_{ww} + \partial_wg_{wu} - \partial_wg_{uw}]\\
    &= 0
\end{align*}
\begin{align*}
    \Gamma^w_{wu}
    &= \frac{1}{2}g^{wu}[\partial_wg_{u} + \partial_ug_{uw} - \partial_ug_{wu}]
    + \frac{1}{2}g^{ww}[\partial_wg_{uw} + \partial_ug_{ww} - \partial_wg_{wu}]\\
    &= 0
\end{align*}
\begin{align*}
    \Gamma^w_{ww}
    &= \frac{1}{2}g^{wu}[\partial_wg_{wu} + \partial_wg_{uw} - \partial_ug_{ww}]
    + \frac{1}{2}g^{ww}[\partial_wg_{ww} + \partial_wg_{ww} - \partial_wg_{ww}]\\
    &= 0
\end{align*}
Therefore the only non-zero Christoffel symbol is $\Gamma^w_{uu}$, then
\begin{align*}
    a^u = \dv{v^u}{t} = \dv{t}(v) = 0
\end{align*}
And
\begin{align*}
    a^w &= \dv{v^w}{t} + \Gamma^w_{uu} (v^u)^2\\
    &= Ab^2v^2\sin(bu) - Ab^2\sin(bu)v^2\\
    &= 0
\end{align*}
Where we used that $v^u = v$ and $v^w = -Abv\cos(bu)$.
\\
Therefore the components of the acceleration in the sinusoidal coordinate
system are zero as well.
\end{proof}

\cleardoublepage
\begin{proof}{\textbf{P17.7}}\\
We know that an observer at rest at $r$ has a four-velocity $\bm{u} = \bm{o}_t$
and the only non-zero component of $\bm{u}$ is the $t$ component given by
\begin{align*}
    u^t = \frac{1}{\sqrt{1 + 2GM/r}}
\end{align*}
The rest of the component are 0 because the observer is at rest.
\\
On the other hand, we know that the acceleration component are given by
\begin{align*}
    a^\mu &= \dv{u^\mu}{\tau} + \Gamma_{\alpha\beta}^\mu u^\alpha u^\beta
\end{align*}
So, for $\mu = t$ we get that
\begin{align*}
    a^t &= \dv{u^t}{\tau} + \Gamma_{rt}^t u^r u^t +  \Gamma_{tr}^t u^t u^r\\
    &= u^r\dv{u^t}{r} + 0\\
    &= 0
\end{align*}
Where we used that $u^r = 0$.
In the case of $\mu = r$
\begin{align*}
    a^r &= \dv{u^r}{\tau} + \Gamma_{\alpha\beta}^r u^\alpha u^\beta\\
    a^r &= 0 + \bigg[\frac{GM}{r^2}\bigg(1 - \frac{2GM}{r}\bigg)\bigg]
    \bigg(\frac{1}{\sqrt{1 - 2GM/r}}\bigg)^2\\
    a^r &= \frac{GM}{r^2}
\end{align*}
For $\mu = \theta$ we see that $u^\phi = u^r = u^\theta = 0$ and the only non-zero
Christoffel symbols are $\Gamma_{r\theta}^\theta, \Gamma_{\theta r}^\theta$
and $\Gamma_{\phi\phi}^\theta$, hence
\begin{align*}
    a^\theta &= 0
\end{align*}
And for $\mu = \phi$ we see that the only non-zero
Christoffel symbols are $\Gamma_{r\phi}^\phi, \Gamma_{\phi r}^\phi,
\Gamma_{\theta\phi}^\phi$ and $\Gamma_{\phi\theta}^\phi$ hence again
\begin{align*}
    a^\phi &= 0
\end{align*}
Finally, we compute the magnitude $a = \sqrt{\bm{a}\cdot\bm{a}}$ as follows
\begin{align*}
    a &= \sqrt{g_{rr}a^ra^r}\\
    &= \sqrt{\bigg(1 - \frac{2GM}{r}\bigg)^{-1}\frac{(GM)^2}{r^4}}\\
    &= \frac{GM}{r^2}\sqrt{\bigg(1 - \frac{2GM}{r}\bigg)^{-1}}
\end{align*}
\end{proof}

\cleardoublepage
\begin{proof}{\textbf{P17.8}}\\
Let $v^\mu = [(1 - 2GM/r), 0, 0, 0]$ we want to compute the components of the
absolute gradient in the Schwartzschild coordinate basis of $v^\mu$.
The absolute gradient components are computed using the following equation
\begin{align*}
    \nabla_\alpha v^\mu = \pdv{v^\mu}{x^\alpha} + \Gamma^\mu_{\alpha\nu}v^\nu
\end{align*}
Let $\mu = t$ then the only non-zero Christoffel symbols with $t$ as a
superscript are $\Gamma_{tr}^t$ and $\Gamma_{rt}^t$ then
\begin{align*}
    \nabla_t v^t
    &= \pdv{v^t}{t} + \Gamma^t_{tt}v^t + \Gamma^t_{tr}v^r
    + \Gamma^t_{t\theta}v^\theta + \Gamma^t_{t\phi}v^\phi = 0
\end{align*}
Where we used that $v^r = 0$ and $\pdv{v^t}{t} = 0$. In the same way
\begin{align*}
    \nabla_r v^t
    &= \pdv{v^t}{r} + \Gamma^t_{rt}v^t + \Gamma^t_{rr}v^r
    + \Gamma^t_{r\theta}v^\theta + \Gamma^t_{r\phi}v^\phi\\
    &= \pdv{v^t}{r} + \Gamma^t_{rt}v^t\\
    &= \frac{2GM}{r^2} + \frac{GM}{r^2}\bigg(1 - \frac{2GM}{r^2}\bigg)^{-1}
    \bigg(1 - \frac{2GM}{r^2}\bigg)\\
    &= \frac{3GM}{r^2}
\end{align*}
Given that $\Gamma^t_{\theta t} = \Gamma^t_{\theta r} = \Gamma^t_{\theta \theta}
= \Gamma^t_{\theta \phi} = 0$ and that
$\Gamma^t_{\phi t} = \Gamma^t_{\phi r} = \Gamma^t_{\phi \theta}
= \Gamma^t_{\phi \phi} = 0$ then
\begin{align*}
    \nabla_\theta v^t = 0
    \quad\text{and}\quad
    \nabla_\phi v^t = 0
\end{align*}
Let now, $\mu = r$ then the only non-zero Christoffel symbols with $r$ as a
superscript are $\Gamma^r_{tt}, \Gamma^r_{rr}, \Gamma^r_{\theta\theta}$ and
$\Gamma^r_{\phi\phi}$ also $\pdv{v^r}{x^\alpha} = 0$ for
$\alpha = t, r, \theta, \phi$ so the absolute gradient components of $v^r$ are
\begin{align*}
    \nabla_t v^r &= \Gamma^r_{tt}v^t = \frac{GM}{r^2}\bigg(1 - \frac{2GM}{r}\bigg)^2\\
    \nabla_r v^r &= \Gamma^r_{rr}v^r = 0\\
    \nabla_\theta v^r &= \Gamma^r_{\theta \theta} v^\theta = 0\\
    \nabla_\phi v^r &= \Gamma^r_{\phi \phi} v^\phi = 0
\end{align*}
For $\mu = \theta$ the only non-zero Christoffel symbols with $\theta$ as
a superscript are $\Gamma^\theta_{r\theta}, \Gamma^\theta_{\theta r}$ and
$\Gamma^\theta_{\phi\phi}$ also $\pdv{v^\theta}{x^\alpha} = 0$ for
$\alpha = t, r, \theta, \phi$ so the absolute gradient components of $v^\theta$
are
\begin{align*}
    \nabla_t v^\theta &= 0\\
    \nabla_r v^\theta &= \Gamma^\theta_{r\theta}v^\theta = 0\\
    \nabla_\theta v^\theta &= \Gamma^\theta_{\theta r} v^r = 0\\
    \nabla_\phi v^\theta &= \Gamma^\theta_{\phi \phi} v^\phi = 0
\end{align*}
Finally for $\mu = \phi$ the only non-zero Christoffel symbols with $\phi$ as
a superscript are $\Gamma^\phi_{r\phi}, \Gamma^\phi_{\phi r}, \Gamma^\phi_{\theta\phi}$
and $\Gamma^\phi_{\phi\theta}$ also $\pdv{v^\phi}{x^\alpha} = 0$ for
$\alpha = t, r, \theta, \phi$ so the absolute gradient components of $v^\phi$
are
\begin{align*}
    \nabla_t v^\phi &= 0\\
    \nabla_r v^\phi &= \Gamma^\phi_{r\phi}v^\phi = 0\\
    \nabla_\theta v^\phi &= \Gamma^\phi_{\theta \phi} v^\phi = 0\\
    \nabla_\phi v^\phi &= \Gamma^\phi_{\phi r} v^r
    + \Gamma^\phi_{\phi \theta} v^\theta = 0
\end{align*}
Therefore the only non-zero components of the absolute gradient of $v^\mu$ are
\begin{align*}
    \nabla_r v^t &= \frac{3GM}{r^2}\\
    \nabla_t v^r &= \frac{GM}{r^2}\bigg(1 - \frac{2GM}{r}\bigg)^2
\end{align*}
\end{proof}
\begin{proof}{\textbf{P17.9}}\\
The absolute gradient of the metric is
\begin{align*}
    \nabla_\alpha g_{\mu\nu}
    &= \partial_\alpha g_{\mu\nu} - \Gamma_{\alpha\mu}^\sigma g_{\sigma\nu}
    - \Gamma_{\alpha\nu}^\sigma g_{\mu\sigma}
\end{align*}
From equation (17.10) we know that
\begin{align*}
    \Gamma_{\mu\nu}^\alpha
    = \frac{1}{2}g^{\alpha\sigma}[\partial_\mu g_{\nu\sigma}
    + \partial_\nu g_{\sigma\mu} - \partial_\sigma g_{\mu\nu}]
\end{align*}
Then
\begin{align*}
    \nabla_\alpha g_{\mu\nu}
    &= \partial_\alpha g_{\mu\nu}
    - \frac{1}{2}g^{\sigma\nu}[\partial_\alpha g_{\mu\nu}
    + \partial_\mu g_{\nu\alpha} - \partial_\nu g_{\alpha\mu}] g_{\sigma\nu}
    - \frac{1}{2}g^{\sigma\mu}[\partial_\alpha g_{\nu\mu}
    + \partial_\nu g_{\mu\alpha} - \partial_\mu g_{\alpha\nu}] g_{\mu\sigma}\\
    &= \partial_\alpha g_{\mu\nu}
    - \frac{1}{2}[\partial_\alpha g_{\mu\nu}
    + \partial_\mu g_{\nu\alpha} - \partial_\nu g_{\alpha\mu}] 
    - \frac{1}{2}[\partial_\alpha g_{\nu\mu}
    + \partial_\nu g_{\mu\alpha} - \partial_\mu g_{\alpha\nu}]\\
    &= \partial_\alpha g_{\mu\nu}
    - \frac{1}{2}[\partial_\alpha g_{\mu\nu}
    + \partial_\mu g_{\nu\alpha} - \partial_\nu g_{\alpha\mu}] 
    - \frac{1}{2}[\partial_\alpha g_{\mu\nu}
    + \partial_\nu g_{\alpha\mu} - \partial_\mu g_{\nu\alpha}]\\
    &= \partial_\alpha g_{\mu\nu}
    - \partial_\alpha g_{\mu\nu}
    - \frac{1}{2}\partial_\mu g_{\nu\alpha} + \frac{1}{2}\partial_\nu g_{\alpha\mu}
    - \frac{1}{2}\partial_\nu g_{\alpha\mu} + \frac{1}{2}\partial_\mu g_{\nu\alpha}\\
    &= 0
\end{align*}
\end{proof}

\cleardoublepage
\begin{proof}{\textbf{P17.10}}\\
Let us consider a coordinate transformation such that
\begin{align*}
    \pdv{x^\alpha}{x'^\mu}
    = \delta_{\mu}^\alpha - \Gamma^\alpha_{\mu\nu,P} (x'^\nu - x'^\nu_P)
\end{align*}
Equation (17.31) state that
\begin{align*}
    g'_{\mu\nu} &= \pdv{x^\alpha}{x'^\mu}\pdv{x^\beta}{x'^\nu}g_{\alpha\beta}
\end{align*}
Then plugging the expression we have for $\partial x^\alpha/x'^\mu$ and
valuing the expression at $P$ give us
\begin{align*}
    g'_{\mu\nu,P}
    &= [\delta_{\mu}^\alpha - \Gamma^\alpha_{\mu\sigma,P} (x'^\sigma_P - x'^\sigma_P)]
    [\delta_{\nu}^\beta - \Gamma^\beta_{\nu\sigma,P} (x'^\sigma_P - x'^\sigma_P)]
    g_{\alpha\beta,P}\\
    &= \delta_{\mu}^\alpha\delta_{\nu}^\beta g_{\alpha\beta, P}\\
    &= g_{\mu\nu, P}
\end{align*}
Then the primed metric at $P$ has the same components as the unprimed metric
at $P$.
\\
Let us compute the derivative of equation (17.31) as follows
\begin{align*}
    \partial_\rho g'_{\mu\nu}
    &= \partial_\rho
    \bigg([\delta_{\mu}^\alpha - \Gamma^\alpha_{\mu\sigma,P} (x'^\sigma - x'^\sigma_P)]
    [\delta_{\nu}^\beta - \Gamma^\beta_{\nu\sigma,P} (x'^\sigma - x'^\sigma_P)]
    g_{\alpha\beta}\bigg)\\
    &= - \Gamma^\alpha_{\mu\rho,P}
    [\delta_{\nu}^\beta - \Gamma^\beta_{\nu\sigma,P} (x'^\sigma_P - x'^\sigma_P)]
    g_{\alpha\beta}\\
    &\quad- [\delta_{\mu}^\alpha - \Gamma^\alpha_{\mu\sigma,P} (x'^\sigma - x'^\sigma_P)]
    \Gamma^\beta_{\nu\rho,P}g_{\alpha\beta}\\
    &\quad + [\delta_{\mu}^\alpha - \Gamma^\alpha_{\mu\sigma,P} (x'^\sigma - x'^\sigma_P)]
    [\delta_{\nu}^\beta - \Gamma^\beta_{\nu\sigma,P} (x'^\sigma - x'^\sigma_P)]
    \partial_\rho g_{\alpha\beta}
\end{align*}
On the other hand, since $\nabla_\rho g_{\alpha\beta} = 0$ from equation (17.40)
we get that
\begin{align*}
    0 &= \partial_\rho g_{\alpha\beta} - \Gamma^\sigma_{\rho\alpha}g_{\sigma\beta}
    - \Gamma^\sigma_{\rho\beta}g_{\alpha\sigma}\\
    \partial_\rho g_{\alpha\beta} &= \Gamma^\sigma_{\rho\alpha}g_{\sigma\beta}
    + \Gamma^\sigma_{\rho\beta}g_{\alpha\sigma}
\end{align*}
If we value this expression at $P$ we get that
\begin{align*}
    \partial_\rho g_{\alpha\beta}\bigg|_P
    &= \Gamma^\sigma_{\rho\alpha,P}g_{\sigma\beta, P}
    + \Gamma^\sigma_{\rho\beta, P}g_{\alpha\sigma, P}
\end{align*}
Also, if we value the
expression we have for $\partial_\rho g'_{\mu\nu}$ at $P$ we get that
\begin{align*}
    \partial_\rho g'_{\mu\nu}\bigg|_P
    &= - \Gamma^\alpha_{\mu\rho,P}\delta_{\nu}^\beta g_{\alpha\beta,P}
    - \delta_{\mu}^\alpha\Gamma^\beta_{\nu\rho,P}g_{\alpha\beta,P}
    + \delta_{\mu}^\alpha \delta_{\nu}^\beta \partial_\rho g_{\alpha\beta}\bigg|_P\\
    &= - \Gamma^\sigma_{\mu\rho,P} g_{\sigma\nu,P}
    - \Gamma^\sigma_{\nu\rho,P}g_{\mu\sigma,P}
    + \delta_{\mu}^\alpha \delta_{\nu}^\beta
    [\Gamma^\sigma_{\rho\alpha,P}g_{\sigma\beta, P}
    + \Gamma^\sigma_{\rho\beta, P}g_{\alpha\sigma, P}]\\
    &= - \Gamma^\sigma_{\mu\rho,P} g_{\sigma\nu,P}
    - \Gamma^\sigma_{\nu\rho,P}g_{\mu\sigma,P}
    + \Gamma^\sigma_{\rho\mu,P}g_{\sigma\nu, P}
    + \Gamma^\sigma_{\rho\nu, P}g_{\mu\sigma, P}\\
    &= - \Gamma^\sigma_{\mu\rho,P}g_{\sigma\nu,P}
    - \Gamma^\sigma_{\nu\rho,P}g_{\mu\sigma,P}
    + \Gamma^\sigma_{\mu\rho,P}g_{\sigma\nu,P}
    + \Gamma^\sigma_{\nu\rho,P}g_{\mu\sigma,P}\\
    &= 0
\end{align*}
Where we used that $\Gamma^\sigma_{\mu\rho} = \Gamma^\sigma_{\rho\mu}$.
\end{proof}
\end{document}