\documentclass[11pt]{article}
\usepackage{amssymb}
\usepackage{amsthm}
\usepackage{enumitem}
\usepackage{amsmath}
\usepackage{bm}
\usepackage{adjustbox}
\usepackage{mathrsfs}
\usepackage{graphicx}
\usepackage{siunitx}
\usepackage[mathscr]{euscript}

\title{\textbf{Solved selected problems of General Relativity - Thomas A. Moore}}
\author{Franco Zacco}
\date{}

\addtolength{\topmargin}{-3cm}
\addtolength{\textheight}{3cm}

\newcommand{\hatr}{\bm{\hat{r}}}
\newcommand{\hatx}{\bm{\hat{x}}}
\newcommand{\haty}{\bm{\hat{y}}}
\newcommand{\hatz}{\bm{\hat{z}}}
\newcommand{\hatth}{\bm{\hat{\theta}}}
\newcommand{\hatphi}{\bm{\hat{\phi}}}
\newcommand{\hatrho}{\bm{\hat{\rho}}}
\theoremstyle{definition}
\newtheorem*{solution*}{Solution}
\renewcommand*{\proofname}{Solution}

\begin{document}
\maketitle
\thispagestyle{empty}

\section*{Chapter 1 - Introduction}

\begin{proof}{\textbf{P1.2}}
\begin{itemize}
    \item [\textbf{a.}] We know that the relativistic Doppler effect is given
    by
    \begin{align*}
        \frac{\lambda}{\lambda_0} = \sqrt{\frac{1-v/c}{1 + v/c}}
    \end{align*}
    Hence the fractional shift in wavelength is
    \begin{align*}
        \frac{\lambda_0 - \lambda}{\lambda_0} = 1 -\frac{\lambda}{\lambda_0} 
        &= 1 -\sqrt{\frac{1-v/c}{1 + v/c}}\\
        &\approx 1 - \bigg(1-\frac{v}{2c}\bigg)\bigg(1-\frac{v}{2c}\bigg)\\
        &\approx 1 - \bigg(1-\frac{v}{2c}\bigg)^2\\
        &\approx \frac{v}{c}
    \end{align*}
    Where we used the binomial approximation twice, from the first line
    to the second and from the third line to the last one.

    On the other hand, we know that the lab is moving upward with respect to
    the IRF (Inertial Reference Frame) with a velocity $v = gt$ where $t$ is 
    the time it takes for the laser to get to the floor hence we can assume
    it to be $t = d/c$ since the lab is moving slowly with respect to
    the laser. Therefore joining this reasoning we have that
    \begin{align*}
        \frac{\lambda_0 - \lambda}{\lambda_0}
        \approx \frac{v}{c} = \frac{gd}{c^2}
    \end{align*}

    \item [\textbf{b.}] The last result also holds in this case because of the
    Equivalence Principle hence numerically we have that the fractional shift
    in wavelength in a lab on the earth's surface is
    \begin{align*}
        \frac{\lambda_0 - \lambda}{\lambda_0}
        \approx \frac{gd}{c^2}
        = \frac{(9.8~m/s^2)(25~m)}{(299792458~m/s)^2}
        = 2.725 \times 10^{-15} 
    \end{align*}

    \item [\textbf{c.}] If we suppose now that the lab is located on
    the surface of a neutron star the magnitude of $g$ using Newton's law
    of universal gravitation is
    \begin{align*}
        g &= G\frac{M}{r^2}
        = (6.674\times 10^{-11}~Nm^2/kg^2)
        \frac{(3.0 \times 10^{30}~kg)}{(12000~m)^2}\\
        &= 1.3904 \times 10^{12}~m/s^2
    \end{align*}
    Therefore
    \begin{align*}
        \frac{\lambda_0 - \lambda}{\lambda_0}
        \approx \frac{gd}{c^2}
        = \frac{(1.3904 \times 10^{12}~m/s^2)(25~m)}{(299792458~m/s)^2}
        = 0.0003867
    \end{align*}

\end{itemize}
\end{proof}
\begin{proof}{\textbf{P1.3}}
\begin{itemize}
    \item [\textbf{a.}] Let us suppose that the laboratory is accelerating in
    deep space with uniform acceleration $\bm{a} = -\bm{g}$ which is equivalent
    to the laboratory being at the earth's surface.

    The laser takes $t = d/c$ to travel the distance $d$
    but in this period the lab moves a distance $y = 1/2gt^2$ perpendicular
    to the laser so we have that the deflection is
    \begin{align*}
        y = \frac{1}{2}gt^2 = \frac{gd^2}{2c^2}
        = \frac{(9.8~m/s^2)\cdot(3.0~m)^2}{2\cdot(299792458~m/s)^2}
        = 4.906 \times 10^{-16}~m
    \end{align*}

    \item [\textbf{b.}] If we suppose that the laboratory sits on the surface
    of a neutron star having a mass of $M = 3.0 \times 10^{30}~kg$ and a radius
    of $R = 12~km$ then the magnitude of $\bm{g}$ is
    \begin{align*}
        g &= G\frac{M}{r^2}
        = (6.674\times 10^{-11}~Nm^2/kg^2)
        \frac{(3.0 \times 10^{30}~kg)}{(12000~m)^2}\\
        &= 1.3904 \times 10^{12}~m/s^2
    \end{align*}
    Therefore using the same equation we have from before the deflection in
    this case is
    \begin{align*}
        y = \frac{gd^2}{2c^2}
        = \frac{(1.3904 \times 10^{12}~m/s^2)\cdot(3.0~m)^2}{2\cdot(299792458~m/s)^2}
        = 6.962 \times 10^{-5}~m
    \end{align*}

\end{itemize}
\end{proof}
\cleardoublepage
\begin{proof}{\textbf{P1.4}}
    First, we want to determine $v_y$ by integrating $a_y$ as follows
    \begin{align*}
        \int_{-\infty}^{\infty} a_y dt &= \int_{-\infty}^{\infty}
        \frac{GM}{r^2} \cos \varphi~ dt\\
        v_y &= \int_{-\infty}^{\infty} \frac{GMR}{r^3}~ dt\\
        v_y &= \int_{-\infty}^{\infty} \frac{GMR}{(x^2 + R^2)^{3/2}}\frac{dx}{c}\\
        v_y &= \frac{GMR}{c}\left[\frac{x}{R^2\sqrt{R^2 + x^2}}\right]_{-\infty}^{\infty}\\
        v_y &= \frac{GMR}{c}\frac{2}{R^2}
    \end{align*}
    Where $\varphi$ is the angle between $r$ and $R$.
    Assuming $\delta$ is small we can approximate it by
    $\delta \approx \sin\delta = v_y/c$ hence we have that
    \begin{align*}
        \delta = v_y/c &= \frac{2GM}{Rc^2}\\
            &= \frac{2(6.67 \times 10^{-11}~Nm^2/kg^2)(1.988 \times 10^{30}~kg)}
            {(6.957 \times 10^8~m)(299792458~m/s)^2}\\
            &= 4.24 \times 10^{-6}
    \end{align*}
\end{proof}
\cleardoublepage
\begin{proof}{\textbf{P1.5}}
    Because of Newtonian mechanics, we know that the acceleration of the ball A
    is
    \begin{align*}
        a_A = \frac{GM}{r_A^2} = \frac{GM}{(R + r)^2}
    \end{align*}
    where $r_A$ is the radius of the earth $R$ plus $r = 22~m$ where the
    frame's center is. In the same way, we have that $a_B$ and $a_C$ are
    \begin{align*}
        a_B = \frac{GM}{(R + 2r)^2} \quad\quad a_C = \frac{GM}{R^2}
    \end{align*}
    Then the relative acceleration $a_B - a_A$ is
    \begin{align*}
        a_B - a_A &= \frac{GM}{R^2(1 + 2r/R)^2} - \frac{GM}{R^2(1 + r/R)^2}\\
            &= \frac{GM}{R^2}\left(\left(1 + \frac{2r}{R}\right)^{-2}
            - \left(1 + \frac{r}{R}\right)^{-2}\right)\\
            &= \frac{GM}{R^2}\left(\left(1 - \frac{4r}{R}\right)
            - \left(1 - \frac{2r}{R}\right)\right)\\
            &= \frac{GM}{R^2}\left(- \frac{2r}{R}\right)\\
            &= -\frac{(6.67 \times 10^{-11}~Nm^2/kg^2)(5.972\times10^{24}~kg)(44~m)}
            {(6380000~m)^3}\\
            &= -6.75 \times 10^{-5}~m/s^2
    \end{align*}
    Assuming $a_A$ has a downward direction then the relative acceleration
    has an upward direction.
    In the same way, we have for $a_C - a_A$ that
    \begin{align*}
        a_C - a_A &= \frac{GM}{R^2} - \frac{GM}{R^2(1 + r/R)^2}\\
            &= \frac{GM}{R^2}\left(1 - \left(1 + \frac{r}{R}\right)^{-2}\right)\\
            &= \frac{GM}{R^2}\left(1 - \left(1 - \frac{2r}{R}\right)\right)\\
            &= \frac{GM}{R^2}\left(\frac{2r}{R}\right)\\
            &= \frac{(6.67 \times 10^{-11}~Nm^2/kg^2)(5.972\times10^{24}~kg)(44~m)}
            {(6380000~m)^3}\\
            &= 6.75 \times 10^{-5}~m/s^2
    \end{align*}
    In this case, the relative acceleration also points downward.
\end{proof}
\end{document}


