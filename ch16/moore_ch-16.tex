\documentclass[11pt]{article}
\usepackage{amssymb}
\usepackage{amsthm}
\usepackage{enumitem}
\usepackage{physics,amsmath}
\usepackage{bm}
\usepackage{adjustbox}
\usepackage{mathrsfs}
\usepackage{graphicx}
\usepackage{siunitx}
\usepackage[mathscr]{euscript}


\title{\textbf{Solved selected problems of General Relativity - Thomas A. Moore}}
\author{Franco Zacco}
\date{}

\addtolength{\topmargin}{-3cm}
\addtolength{\textheight}{3cm}

\newcommand{\hatr}{\bm{\hat{r}}}
\newcommand{\hatn}{\bm{\hat{n}}}
\newcommand{\hatx}{\bm{\hat{x}}}
\newcommand{\haty}{\bm{\hat{y}}}
\newcommand{\hatz}{\bm{\hat{z}}}
\newcommand{\hatth}{\bm{\hat{\theta}}}
\newcommand{\hatphi}{\bm{\hat{\phi}}}
\newcommand{\hatrho}{\bm{\hat{\rho}}}
\newcommand{\er}{\bm{e}_r}
\newcommand{\etht}{\bm{e}_\theta}
\newcommand{\sci}[1]{\times 10^{#1}}

\theoremstyle{definition}
\newtheorem*{solution*}{Solution}
\renewcommand*{\proofname}{Solution}

\begin{document}
\maketitle
\thispagestyle{empty}

\section*{Chapter 16 - Blackhole Thermodynamics}

\begin{proof}{\textbf{BOX 16.1} - Exercise 16.1.1.}\\
From BOX 14.2 we know a free-falling object falling from $2GM + \varepsilon$
has
\begin{align*}
    e^2 = 1 - \frac{2GM}{2GM + \varepsilon}
\end{align*}
Then replacing into the equation (16.12) where $l =0$ we get that
\begin{align*}
    \dv{r}{\tau} &= \sqrt{
        \bigg(1 - \frac{2GM}{2GM + \varepsilon}\bigg) - 
        \bigg(1 - \frac{2GM}{r}\bigg)
    }\\
    \dv{r}{\tau} &= \sqrt{\frac{2GM}{r}-\frac{2GM}{2GM + \varepsilon}}
\end{align*}
Now, integrating between $2GM + \varepsilon$ and $2GM$ we see that
\begin{align*}
    \Delta \tau = -\int_{2GM + \varepsilon}^{2GM}
    \frac{dr}{\sqrt{\frac{2GM}{r}-\frac{2GM}{2GM + \varepsilon}}}
\end{align*}
Let $\rho = r - 2GM$ then $d\rho = dr$ and inverting the integral limits we
have that
\begin{align*}
    \Delta \tau = +\int^{\varepsilon}_{0}
    \frac{d\rho}{\sqrt{\frac{2GM}{2GM + \rho}-\frac{2GM}{2GM + \varepsilon}}}
\end{align*}
\end{proof}

\cleardoublepage
\begin{proof}{\textbf{BOX 16.1} - Exercise 16.1.2.}\\
Let us write equation (16.13) as follows
\begin{align*}
    \Delta \tau &= \int^{\varepsilon}_{0}
    \frac{d\rho}{\sqrt{\frac{2GM}{2GM + \rho}-\frac{2GM}{2GM + \varepsilon}}}\\
    &= \int^{\varepsilon}_{0}
    \frac{d\rho}{\sqrt{\frac{1}{1 + \rho/2GM}-\frac{1}{1 + \varepsilon/2GM}}}\\
    &= \int^{\varepsilon}_{0}
    \frac{d\rho}{\sqrt{(1 + \rho/2GM)^{-1}- (1 + \varepsilon/2GM)^{-1}}} 
\end{align*}
Then if we let $\varepsilon \ll 2GM$ applying the binomial approximation gives
us
\begin{align*}
    \Delta \tau &\approx \int^{\varepsilon}_{0}
    \frac{d\rho}{\sqrt{(1 - \rho/2GM)- (1 - \varepsilon/2GM)}}\\
    &\approx \sqrt{2GM}\int^{\varepsilon}_{0}\frac{d\rho}{\sqrt{\varepsilon - \rho}}\\
    &\approx \sqrt{2GM}\bigg[-2\sqrt{\varepsilon - \rho}\bigg]_0^\varepsilon\\
    &\approx 2\sqrt{2GM\varepsilon}
\end{align*}
\end{proof}

\cleardoublepage
\begin{proof}{\textbf{BOX 16.2} - Exercise 16.2.1.}\\
Given that the spatial components of the frame's four-velocity will be zero
at the time the particle-antiparticle pair is formed, then the only nonzero
component is $u^t$ then
\begin{align*}
    \bm{u}\cdot\bm{u} = g_{\mu\nu}u^{\mu}u^{\nu} &= -1\\
    -\bigg(1 - \frac{2GM}{r}\bigg) (u^t)^2 &= -1\\
    u^t &= \frac{1}{\sqrt{1 - \frac{2GM}{r}}}
\end{align*}
\end{proof}
\begin{proof}{\textbf{BOX 16.2} - Exercise 16.2.2.}\\
We compute $E = -\bm{o}_t\cdot\bm{p} = -\bm{u}\cdot\bm{p}$ as follows
\begin{align*}
    E &= -g_{\mu\nu}u^\mu p^\nu
    = \bigg(1 - \frac{2GM}{r}\bigg)\frac{1}{\sqrt{1 - \frac{2GM}{r}}} m\dv{t}{\tau}
    = \sqrt{1 - \frac{2GM}{r}} m\dv{t}{\tau}
\end{align*}
Where we used again that $u^r=u^\theta=u^\phi = 0$ initially.
\end{proof}
\begin{proof}{\textbf{BOX 16.2} - Exercise 16.2.3.}\\
We know that
\begin{align*}
    e = \bigg(1 - \frac{2GM}{r}\bigg)\dv{t}{\tau}
\end{align*}
Then mutiplying the equation by $m$ we get that
\begin{align*}
    me &= \bigg(1 - \frac{2GM}{r}\bigg)m\dv{t}{\tau}\\
    E_\infty &= \bigg(1 - \frac{2GM}{r}\bigg)\frac{E}{\sqrt{1 - \frac{2GM}{r}}}\\
    E_\infty &= \sqrt{1 - \frac{2GM}{r}}E
\end{align*}
Hence
\begin{align*}
    E &= \frac{E_\infty}{\sqrt{1 - \frac{2GM}{r}}}
\end{align*}
Where $E_\infty$ is the particle's energy at infinity.

\end{proof}

\cleardoublepage
\begin{proof}{\textbf{BOX 16.3} - Exercise 16.3.1.}\\
We know that in SI $k_B = 1.3807 \times 10^{-23} J/K$ and
$\hbar = 1.0546 \times 10^{-34} Js$ then

\begin{align*}
\frac{\hbar}{8\pi k_B GM_{\odot}}
&= \frac{1.0546 \times 10^{-34}~Js \cdot (299800000~m/1~s)}
{8 \pi \cdot 1.3807 \times 10^{-23} J/K \cdot 1477~m}
= 6.168\times 10^{-8} K
\end{align*}
\end{proof}
\begin{proof}{\textbf{BOX 16.4} - Exercise 16.4.1.}\\
We know that a black hole's mass-energy $M$ is radiated away following the
Stefan-Boltzmann formula
\begin{align*}
    \dv{M}{t} = A\sigma T^4
\end{align*}
Hence, we need to integrate as follows
\begin{align*}
    \int_0^{\tau} dt &= \frac{1}{\sigma} \int_0^M \frac{dM}{A T^4}
\end{align*}
So, using that $T = \hbar/8\pi k_B GM$ and $A = 4\pi (2GM)^2$ we get that
\begin{align*}
    \tau &=  \frac{(8\pi k_B G)^4}{16\pi G^2\sigma \hbar^4}\int_0^M M^2 dM
    = \frac{8^4\pi^3 k_B^4 G^2}{16\sigma \hbar^4}\frac{M^3}{3}
    = \frac{256\pi^3 k_B^4}{3G\sigma \hbar^4}(GM)^3
\end{align*}
\end{proof}
\begin{proof}{\textbf{BOX 16.4} - Exercise 16.4.2.}\\
Using that $k_B = 1.536\times10^{-40}~kg/K$, $\hbar = 3.518\times 10^{-43}~kgm$, 
and $\sigma = 2.105\times 10^{-33}~kgm^{-3}K^{-4}$ we have that
\begin{align*}
    \tau &= \frac{256\pi^3 (1.536\times10^{-40})^4}
    {3 \cdot (7.426 \times 10^{-28}) (2.105\times 10^{-33}) (3.518\times 10^{-43})^4}(1477)^3
    \bigg(\frac{M}{M_{\odot}}\bigg)^3\\
    &= 1.982 \times 10^{83}~m\cdot \frac{1~y}{9.461\times 10^{15}~m} \cdot
    \bigg(\frac{M}{M_{\odot}}\bigg)^3\\
    &= (2.095 \times 10^{67}~y)\bigg(\frac{M}{M_{\odot}}\bigg)^3
\end{align*}
\end{proof}

\cleardoublepage
\begin{proof}{\textbf{P16.1}}
From equation 16.6 we know that the relativistic energy at infinity is
\begin{align*}
    E_\infty = \frac{\hbar}{4GM}
\end{align*}
If $M$ is a solar mass then we get that
\begin{align*}
    E_\infty = \frac{3.518\sci{-43}~kg\cdot m}{4 \cdot 1477~m} = 5.95\sci{-47}~kg
    \approx 6\sci{-47}~kg
\end{align*}
Or in eV
$$E_\infty = 5.95\sci{-47}~kg \cdot \frac{1~eV}{1.782\sci{-36}~kg}
= 3.33\sci{-11}~eV$$
\end{proof}

\cleardoublepage
\begin{proof}{\textbf{P16.2}}
\begin{itemize}
\item [\textbf{a.}] Using equation 16.9 we know that a blackhole survives
for
$$\tau_{\text{life}} = (2.095 \times 10^{67}~y)\bigg(\frac{M}{M_{\odot}}\bigg)^3$$
So if a balckhole is just evaporating today, then we can compute how massive it
was in the beginning as follows
\begin{align*}
    (2.095 \times 10^{67}~y)\bigg(\frac{M}{M_{\odot}}\bigg)^3 &= 13.7\sci{9}~y\\
    \frac{M}{M_{\odot}} &= \sqrt[3]{\frac{13.7\sci{9}}{2.095 \times 10^{67}}}\\
    M &= 8.679\sci{-20}M_{\odot}
\end{align*}
So the black hole was $8.679\sci{-20}$ solar masses or $173.58\sci{9} ~kg$.

\item [\textbf{b.}] We want to know now how much energy a black hole releases
in the last second of its life. So, from the Stefan-Boltzmann formula we know
that
\begin{align*}
    -\dv{M}{t} = \dv{E_{rad}}{t} = A\sigma T^4
\end{align*}
We need to integrate it to get the energy radiated but we integrate with
respect to the mass. So we get again that 
$$1~s = (2.095 \times 10^{67}~y)\bigg(\frac{M}{M_{\odot}}\bigg)^3$$
Where we inegrated $t$ from 0 to 1 second.
Then solving for $M = E_{rad}$ we see that
\begin{align*}
    E_{rad} &= M = \sqrt[3]{\frac{3.17098\sci{-8}~y}{2.095 \times 10^{67}~y}}M_\odot\\
    &= 229632.39~kg \cdot \frac{1~J}{1.1126\sci{-17}~kg}\\
    &= 2.06\sci{22}~J
\end{align*}
This is $51.5\sci{6}$ times an atomic bomb.
\end{itemize}
\end{proof}

\cleardoublepage
\begin{proof}{\textbf{P16.3}}
A black hole of $1~TeV$ mass or $1.782\sci{-24}~kg$ will have a lifetime of
\begin{align*}
    \tau &= (2.095 \times 10^{67}~y)\bigg(\frac{1.782\sci{-24}~kg}{2\sci{30}~kg}\bigg)^3\\
    &= 2.095 \times 10^{67} \cdot 7.073\sci{-163}~y\\
    &= 1.481\sci{-95}~y = 4.67\sci{-88}~s
\end{align*}
\end{proof}

\cleardoublepage
\begin{proof}{\textbf{P16.4}}
\begin{itemize}
\item [\textbf{a.}] From equation 14.3 we know that the physical distance from
$r = R$ to $r = 2GM$ is
\begin{align*}
    \Delta s = R\sqrt{1 - \frac{2GM}{r}} + 2GM\tanh^{-1}\sqrt{1 - \frac{2GM}{r}}
\end{align*}
We know that the box can be lowered down at most a physical distance from the
event horizon of $\Delta s = \lambda/2$, also, applying the approximation
$\tanh^{-1}x \approx x$ and assuming that $R \approx 2GM$ we get that
\begin{align*}
    \frac{\lambda}{2}
    &\approx 2GM\sqrt{1 - \frac{2GM}{r}} + 2GM\sqrt{1 - \frac{2GM}{r}}\\
    \frac{\lambda}{2} 
    &\approx 4GM\sqrt{1 - \frac{2GM}{r}}\\
    \bigg(\frac{\lambda}{2}\bigg)^2
    &\approx (4GM)^2\bigg(1 - \frac{2GM}{r}\bigg)
\end{align*}
\item [\textbf{b.}] From equation 10.8 we know that
\begin{align*}
    \frac{1}{2}(e^2 - 1)
    &= \frac{1}{2}\bigg(\dv{r}{\tau}\bigg)^2 - \frac{GM}{r} + \frac{l^2}{2r^2}
    - \frac{GMl^2}{r^3}
\end{align*}
Using that $dr/d\tau = 0$ and $l = 0$ since the object is at rest
we get that
\begin{align*}
    e &= \sqrt{1 - \frac{2GM}{r}}
\end{align*}
Also, from part (a) we see that
\begin{align*}
    \bigg(\frac{\lambda}{8GM}\bigg)^2 &\approx \bigg(1 - \frac{2GM}{r}\bigg)\\
    \frac{\lambda}{8GM} &\approx \sqrt{1 - \frac{2GM}{r}}
\end{align*}
Therefore
\begin{align*}
    e &\approx \frac{\lambda}{8GM}
\end{align*}
\item [\textbf{c.}] If we consider that at the beginning the energy per unit
of mass of the box is $e = 1$, close to the event horizon is $e$ and that 
the difference $1 - e$ has become work done on the winch, then the efficiency
must be $1 - e$ since all the energy has become work.

Then must be that $e = T_C/T_H$ but we know that $\lambda \approx \hbar / k_B T_H$
so
\begin{align*}
    e = \frac{\lambda}{8GM} = \frac{\lambda k_B T_C}{\hbar} = \frac{T_C}{T_H}
\end{align*}
Therefore $T_C$ must be $T_C = \hbar/8GM k_B$.
\end{itemize}
\end{proof}

\cleardoublepage
\begin{proof}{\textbf{P16.5}}
According to equation (12.15) observers at $r = R$ will see photons with
energy $E_{obs}$
\begin{align*}
    E_{obs} = \frac{E_\infty}{\sqrt{1 - 2GM/R}}
\end{align*}
Where $E_\infty$ is the photon's energy at infinity. So an observer at $r = R$
will consider the temperature of the black hole to be higher by a factor of
$1/\sqrt{1 - 2GM/R}$ than that of an observer at infinity.
\\
As $R \to 2GM$ the temperature tends to infinity.
\\
Let us note that
\begin{align*}
    k_B T = E_{obs} = \frac{E_\infty}{\sqrt{1 - 2GM/R}}
    = \frac{k_B T_\infty}{\sqrt{1 - 2GM/R}}
\end{align*}
Then 
\begin{align*}
    T = \frac{T_\infty}{\sqrt{1 - 2GM/R}}
\end{align*}
So since we are considering a solar-mass black hole we have that
$T_\infty = 6.17 \times 10^{-8}~K$. Therefore to observe a temperature of
$T = 300~K$ we need to be at the following distance (in $R$) from the event
horizon
\begin{align*}
    R - 2GM &= \frac{2GM}{1 - (T_\infty/T)^2} - 2GM\\
    &= 2\cdot 1477 \bigg(\frac{1}{1 - (6.17 \times 10^{-8}/300)^2} - 1\bigg)\\
    &= 1.249 \sci{-16} ~m
\end{align*}

\end{proof}

\cleardoublepage
\begin{proof}{\textbf{P16.6}}
We know from equation (16.18) that
\begin{align*}
    \dv{M}{t} = -A\sigma T^4
    = -\frac{4\pi(4G^2M^2)\sigma\hbar^4}{4096\pi^4 k_B^4 G^4M^4}
    = -\frac{\sigma\hbar^4}{256\pi^3 k_B^4 G^2M^2}
\end{align*}
Where we replaced the values we have for $A$ and $T$. Then by integration we
get that
\begin{align*}
    \int_{M_0}^M M^2~dM
    &= -\frac{\sigma\hbar^4}{256\pi^3 k_B^4 G^2}\int_0^t~dt\\
    \frac{M^3}{3} - \frac{M_0^3}{3}
    &= -\frac{\sigma\hbar^4}{256\pi^3 k_B^4 G^2}t\\
    M^3 &= -\frac{3\sigma\hbar^4}{256\pi^3 k_B^4 G^2}t + M_0^3
\end{align*}
But we know that the lifetime of a black hole is
\begin{align*}
    t_0 = \frac{256\pi^3k_B^4G^2}{3\sigma \hbar^4} M_0^3
\end{align*}
Therefore
\begin{align*}
    M^3 &= -\frac{M_0^3}{t_0}t + M_0^3\\
    M^3 &= M_0^3\bigg(1-\frac{t}{t_0}\bigg)\\
    M &= M_0\sqrt[3]{1-\frac{t}{t_0}}
\end{align*}
Below we leave a plot of $M/M_0$ as a function of $t/t_0$
\begin{center}
    \includegraphics[scale=0.25]{ch16-p16.6.png}
\end{center}
\end{proof}

\cleardoublepage
\begin{proof}{\textbf{P16.7}}
Let two black holes have masses $M_1$ and $M_2$ then the entropy of each is
\begin{align*}
    S_1 = \frac{4\pi k_B G}{\hbar} M_1^2
    \quad\text{and}\quad
    S_2 = \frac{4\pi k_B G}{\hbar} M_2^2
\end{align*}
So the total entropy is
\begin{align*}
    S_1 + S_2 &= \frac{4\pi k_B G}{\hbar} M_1^2 + \frac{4\pi k_B G}{\hbar} M_2^2
    = \frac{4\pi k_B G}{\hbar} (M_1^2 + M_2^2)
\end{align*}
Now, if we consider they merge toghether to a new black hole of mass $M_1 + M_2$
then the entropy of this new black hole would be
\begin{align*}
    S_M &= \frac{4\pi k_B G}{\hbar} (M_1 + M_2)^2
\end{align*}
And since $M_1^2 + M_2^2 < M_1^2 + 2M_1M_2 + M_2^2 = (M_1 + M_2)^2$
we see that
\begin{align*}
    S_1 + S_2 < S_M 
\end{align*}
\end{proof}

\cleardoublepage
\begin{proof}{\textbf{P16.8}}
Let us consider a system composed of a thermal reservoir and a black hole, then
the entropy of the reservoir is
\begin{align*}
    S_R = \frac{U_{tot} - M}{T_R} + C
\end{align*}
Where $U_{tot}$ is the total conserved energy of the system, $M$ is the mass
of the black hole, $T_R$ is the constant temperature of the reservoir and $C$
is a constant.
\\
Also, the entropy of the black hole we know it is 
\begin{align*}
    S_{BH} = \frac{4\pi k_B G}{\hbar} M^2
\end{align*}
So the total entropy of the system is
\begin{align*}
    S_{tot} = S_{BH} + S_R
    = \frac{4\pi k_B G}{\hbar} M^2 -\frac{M}{T_R} + \bigg(\frac{U_{tot}}{T_R} + C\bigg)
\end{align*}
We know that a stable equilibrium is when energy is distributed in such a way
that the total entropy of the system is a local maximum.
\\
In this case, $S_{tot}$ is a quadratic equation of $M$ and since the term
involving $M^2$ is positive then the branches point upward, hence the only
extremum point point is a minimum.
\\
Also, $S_{tot}$ cannot take values such that $M < 0$ and $M > U_{tot}$ so the
only local maximum of the quadratic equation can happen for $M = 0$ and
$M = U_{tot}$ i.e. the extremes.
\end{proof}

\cleardoublepage
\begin{proof}{\textbf{P16.9}}
Let us compute the entropy $S$ of a solar-mass black hole as follows
\begin{align*}
    S = \frac{4\pi k_B GM^2}{\hbar}
    = \frac{4\pi \cdot 1.536\sci{-40} \cdot 1477 \cdot 2\sci{30}}{3.518\sci{-43}} 
    = 1.621\sci{37}~kg/K
\end{align*}
On the other hand, if we consider that the black hole was formed by pure
ionized hydrogen (protons), then the number of particles forming it is
\begin{align*}
    N = \frac{M}{m_p}
    = \frac{2\sci{30}}{1.672\sci{-27}}
    = 1.196\sci{57}
\end{align*}
Therefore the ratio $S/Nk_B$ for a solar-mass blackhole is 
\begin{align*}
    \frac{S}{Nk_B} = \frac{1.621\sci{37}}{1.196\sci{57} \cdot 1.536\sci{-40}}
    = 8.823\sci{19}
\end{align*}
In comparison to the normal matter, where the ratio is $S/Nk_B = 1$, we see
that it is 19 orders of magnitude bigger.
\end{proof}
\end{document}