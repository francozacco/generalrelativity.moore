\documentclass[11pt]{article}
\usepackage{amssymb}
\usepackage{amsthm}
\usepackage{enumitem}
\usepackage{physics,amsmath}
\usepackage{bm}
\usepackage{adjustbox}
\usepackage{mathrsfs}
\usepackage{graphicx}
\usepackage{siunitx}
\usepackage[mathscr]{euscript}


\title{\textbf{Solved selected problems of General Relativity - Thomas A. Moore}}
\author{Franco Zacco}
\date{}

\addtolength{\topmargin}{-3cm}
\addtolength{\textheight}{3cm}

\newcommand{\hatr}{\bm{\hat{r}}}
\newcommand{\hatn}{\bm{\hat{n}}}
\newcommand{\hatx}{\bm{\hat{x}}}
\newcommand{\haty}{\bm{\hat{y}}}
\newcommand{\hatz}{\bm{\hat{z}}}
\newcommand{\hatth}{\bm{\hat{\theta}}}
\newcommand{\hatphi}{\bm{\hat{\phi}}}
\newcommand{\hatrho}{\bm{\hat{\rho}}}
\newcommand{\er}{\bm{e}_r}
\newcommand{\etht}{\bm{e}_\theta}

\theoremstyle{definition}
\newtheorem*{solution*}{Solution}
\renewcommand*{\proofname}{Solution}

\begin{document}
\maketitle
\thispagestyle{empty}

\section*{Chapter 7 - Maxwell's Equations}

\begin{proof}{\textbf{BOX 7.1} - Exercise 7.1.1.}
    In the same way, as we did for the $x$ direction the flux going through 
    the face which has a unit vector $\hatn$ pointing to the $-y$ direction is
    $$\bm{\bar{E}}(x, y - \frac{1}{2}\Delta y, z) \cdot \hatn \Delta x \Delta z
    = -E_y(x, y - \frac{1}{2}\Delta y, z)\Delta x \Delta z$$
    Similarly, the flux through the face with a unit vector pointing to the $+y$
    direction gives us
    $$\bm{\bar{E}}(x, y + \frac{1}{2}\Delta y, z) \cdot \hatn \Delta x \Delta z
    = E_y(x, y + \frac{1}{2}\Delta y, z)\Delta x \Delta z$$
    Therefore the net flux through these two faces is
    \begin{align*}
        [E_y(x, y + \frac{1}{2}\Delta y, z)&-E_y(x, y - \frac{1}{2}\Delta y, z)]
        \Delta x \Delta z
        =\\
        &=\bigg[\frac{E_y(x, y + \frac{1}{2}\Delta y, z)
         - E_y(x, y - \frac{1}{2}\Delta y, z)}{\Delta y}\bigg]
        \Delta x \Delta y\Delta z\\
        &\approx \partialderivative{E_y}{y} \Delta x \Delta y\Delta z
    \end{align*}
    On the other hand, by the same means, we can get that net flux in the faces
    perpendicular to the $z$ axis is
    \begin{align*}
        [E_z(x, y, z+ \frac{1}{2}\Delta z)&-E_z(x, y, z - \frac{1}{2}\Delta z)]
        \Delta x \Delta y
        =\\
        &=\bigg[\frac{E_z(x, y, z + \frac{1}{2}\Delta z)
         - E_z(x, y, z- \frac{1}{2}\Delta z)}{\Delta z}\bigg]
        \Delta x \Delta y\Delta z\\
        &\approx \partialderivative{E_z}{z} \Delta x \Delta y\Delta z
    \end{align*}
\end{proof}
\cleardoublepage
\begin{proof}{\textbf{BOX 7.2} - Exercise 7.2.1.}
    We know that $-m^2 = p_\mu p^\mu = p^\mu \eta_{\mu\nu}p^\nu$ hence
    by derivating with respect to $\tau$ and applying the product rule
    we get that
    \begin{align*}
        \frac{d}{d\tau}(-m^2) &= \frac{d}{d\tau}(p^\mu \eta_{\mu\nu}p^\nu)\\
        0 &= \eta_{\mu\nu}p^\nu \frac{dp^\mu}{d\tau}
        + p^\mu \frac{d(\eta_{\mu\nu}p^\nu)}{d\tau}\\
        0 &= \eta_{\mu\nu}p^\nu \frac{dp^\mu}{d\tau}
        + \eta_{\nu\mu}p^\mu \frac{dp^\nu}{d\tau}\\
        0 &= 2\eta_{\mu\nu}p^\nu \frac{dp^\mu}{d\tau}\\
        0 &= 2p_\mu \frac{dp^\mu}{d\tau} 
    \end{align*}
    Where we used that the metric tensor $\eta_{\mu\nu}$ is symmetric.
\end{proof}
\begin{proof}{\textbf{BOX 7.3} - Exercise 7.3.1.}
    Let $B_\mu$ be an arbitrary covector, then let us raise the index
    in cartesian coordinates as follows
    \begin{align*}
        B^t &= \eta^{tt}B_t + \eta^{tx}B_x + \eta^{ty}B_y + \eta^{tz}B_z
            = (-1)\cdot B_t + 0 + 0 + 0 = -B_t\\
        B^x &= \eta^{xt}B_t + \eta^{xx}B_x + \eta^{xy}B_y + \eta^{xz}B_z
            = 0 + 1\cdot B_x + 0 + 0 = B_x\\
        B^y &= \eta^{yt}B_t + \eta^{yx}B_x + \eta^{yy}B_y + \eta^{yz}B_z
            = 0 + 0 + 1\cdot B_y + 0 = B_y\\
        B^x &= \eta^{zt}B_t + \eta^{zx}B_x + \eta^{zy}B_y + \eta^{zz}B_z
            = 0 + 0 + 0 + 1\cdot B_z = B_z
        \end{align*}
\end{proof}
\cleardoublepage
\begin{proof}{\textbf{BOX 7.4} - Exercise 7.4.1.}
    In the same way, as we did for the $x$ direction the total amount of 
    charge that moves out of the box through the front face during a 
    time interval $\Delta t$ is
    \begin{align*}
        \Delta q_{\text{front}} &\approx
        - \rho(x, y - \frac{1}{2}\Delta y, z)
        v_y(x, y - \frac{1}{2}\Delta y, z)\Delta t \Delta x \Delta z\\
        &\approx - J^y(x, y - \frac{1}{2}\Delta y, z)\Delta t \Delta x \Delta z
    \end{align*}
    In the same way, the amount of charge flowing out of the back face
    is $ \Delta q_{\text{back}}
    \approx J^y(x, y + \frac{1}{2}\Delta y, z)\Delta t \Delta x \Delta z$
    and hence the net amount of charge flowing out of these two faces during
    $\Delta t$ will be
    \begin{align*}
        \Delta q_{\text{back}} + \Delta q_{\text{front}}
        &\approx [J^y(x, y + \frac{1}{2}\Delta y, z)
        - J^y(x, y - \frac{1}{2}\Delta y, z)]\Delta t \Delta x \Delta z\\
        \frac{\Delta q_{\text{back}} + \Delta q_{\text{front}}
        }{\Delta t\Delta x \Delta y\Delta z}
        &\approx \frac{J^y(x, y + \frac{1}{2}\Delta y, z) 
        - J^y(x, y - \frac{1}{2}\Delta y, z)}{\Delta y}
    \end{align*}
    Note that this approximations become exact in the limit when
    $\Delta x, \Delta y, \Delta z$ and $\Delta t$ goes to zero, hence by the 
    definition of partial derivative we get that
    \begin{align*}
    \lim_{\Delta y \to 0} \frac{J^y(x, y + \frac{1}{2}\Delta y, z) 
        - J^y(x, y - \frac{1}{2}\Delta y, z)}{\Delta y} 
    = \partialderivative{J^y}{y}
    \end{align*}

    On the other hand, by the same means, we can get the net amount of charge
    flowing out of the top and bottom faces during $\Delta t$ as
    \begin{align*}
        \Delta q_{\text{top}} + \Delta q_{\text{bottom}}
        &\approx [J^z(x, y, z + \frac{1}{2}\Delta z)
        - J^z(x, y, z - \frac{1}{2}\Delta z)]\Delta t \Delta x \Delta y\\
        \frac{\Delta q_{\text{top}} + \Delta q_{\text{bottom}}
        }{\Delta t\Delta x \Delta y\Delta z}
        &\approx \frac{J^z(x, y, z + \frac{1}{2}\Delta z) 
        - J^z(x, y, z - \frac{1}{2}\Delta z)}{\Delta z}
    \end{align*}
    And as $\Delta z$ goes to 0 we get that
    \begin{align*}
        \lim_{\Delta z \to 0} \frac{J^z(x, y, z + \frac{1}{2}\Delta z) 
            - J^y(x, y, z - \frac{1}{2}\Delta z)}{\Delta z} 
        = \partialderivative{J^z}{z}
        \end{align*}
\end{proof}
\cleardoublepage
\begin{proof}{\textbf{BOX 7.4} - Exercise 7.4.2.}
    Combining expressions 7.29 and 7.30 we get that
    \begin{align*}
        &\frac{\Delta q_{\text{left}} + \Delta q_{\text{right}}
        }{\Delta t\Delta x \Delta y\Delta z}
        +
        \frac{\Delta q_{\text{back}} + \Delta q_{\text{front}}
        }{\Delta t\Delta x \Delta y\Delta z}
        +
        \frac{\Delta q_{\text{top}} + \Delta q_{\text{bottom}}
        }{\Delta t\Delta x \Delta y\Delta z} \approx\\
        &\qquad\approx \frac{J^x(x + \frac{1}{2}\Delta x, y, z) 
        - J^x(x - \frac{1}{2}\Delta x, y, z)}{\Delta x} +\\
        &\qquad+ \frac{J^y(x, y + \frac{1}{2}\Delta y, z) 
        - J^y(x, y - \frac{1}{2}\Delta y, z)}{\Delta y} +\\
        &\qquad+ \frac{J^z(x, y, z + \frac{1}{2}\Delta z) 
        - J^z(x, y, z - \frac{1}{2}\Delta z)}{\Delta z}
    \end{align*}
    Hence by applying the limit to both sides and using equation 7.32 we have that
    \begin{align*}
        -\partialderivative{\rho}{t} = \partialderivative{J^x}{x}
        + \partialderivative{J^y}{y} + \partialderivative{J^z}{z}
    \end{align*}
    But also we know that $\rho = J^t$ therefore
    \begin{align*}
        \partialderivative{J^t}{t} + \partialderivative{J^x}{x}
         + \partialderivative{J^y}{y} + \partialderivative{J^z}{z} = 0
    \end{align*}
    Or using Einstein notation $\partial_\mu J^\mu = 0$.
\end{proof}
\begin{proof}{\textbf{BOX 7.5} - Exercise 7.5.1.}
    We know that $F^{\mu\nu} = -F^{\nu\mu}$ then we have that
    \begin{align*}
        \partial_\mu\partial_\nu F^{\mu\nu}
        = \partial_\mu\partial_\nu (-F^{\nu\mu})
        = -\partial_\nu\partial_\mu F^{\nu\mu}
        = -\partial_\mu\partial_\nu F^{\mu\nu}
    \end{align*}
    where we used that the order of partial derivatives is irrelevant
    and in the last equality, we renamed the variables
    $\nu \to \mu$ and $\mu \to \nu$.
    This implies that $\partial_\mu\partial_\nu F^{\mu\nu} = 0$.
\end{proof}
\begin{proof}{\textbf{BOX 7.6} - Exercise 7.6.1.}
    Let $\vec{B} = \vec\nabla \times \vec A$ then by solving the cross product
    we have that
    \begin{align*}
        \vec{B} = \vec\nabla \times \vec A
        &= \begin{vmatrix}
            \hatx & \haty & \hatz\\
            \partial_x & \partial_y & \partial_z\\
            A^x & A^y & A^z
        \end{vmatrix}\\
        &= (\partial_yA^z -  \partial_zA^y)\hatx
        + (\partial_zA^x - \partial_xA^z)\haty
        + (\partial_xA^y - \partial_yA^x)\haty\\
        &= \bigg(
            \partialderivative{A^z}{y} -  \partialderivative{A^y}{z}
        \bigg)\hatx
        + \bigg(
            \partialderivative{A^x}{z} - \partialderivative{A^z}{x}
        \bigg)\haty
        + \bigg(
            \partialderivative{A^y}{x} - \partialderivative{A^x}{y}
        \bigg)\haty
    \end{align*}
\end{proof}
\cleardoublepage
\begin{proof}{\textbf{BOX 7.6} - Exercise 7.6.2.}
    Let $F^{\mu\nu} = \partial^\mu A^\nu - \partial^\nu A^\mu$ then for
    $\mu = t$ and $\nu = x$ we have that
    \begin{align*}
        F^{tx} &= \partial^t A^x - \partial^x A^t\\
        E_x &= -\partialderivative{A^x}{t} - \partialderivative{A^t}{x}
    \end{align*}
    which is the $x$ component of equation 7.35.
    For $\mu = \nu = t$ we have that
    \begin{align*}
        F^{tt} = \partial^t A^t - \partial^t A^t = 0
    \end{align*}
    which is the correct component of the field tensor.
\end{proof}
\begin{proof}{\textbf{BOX 7.7} - Exercise 7.7.1.}
    Let $F^{\mu\nu} = \partial^\mu A^\nu - \partial^\nu A^\mu$ then by replacing 
    we have that
    \begin{align*}
        \partial^\alpha F^{\mu\nu} + \partial^\nu F^{\alpha\mu}
        + \partial^\mu F^{\nu\alpha}
        %
        &= \partial^\alpha (\partial^\mu A^\nu - \partial^\nu A^\mu)
        + \partial^\nu (\partial^\alpha A^\mu - \partial^\mu A^\alpha) +\\
        &\quad+ \partial^\mu (\partial^\nu A^\alpha - \partial^\alpha A^\nu)\\
        %
        &= \partial^\alpha\partial^\mu A^\nu - \partial^\alpha\partial^\nu A^\mu
        + \partial^\nu \partial^\alpha A^\mu - \partial^\nu \partial^\mu A^\alpha +\\
        &\quad+ \partial^\mu\partial^\nu A^\alpha - \partial^\mu\partial^\alpha A^\nu\\
        %
        &= (\partial^\alpha\partial^\mu A^\nu
        - \partial^\alpha\partial^\mu A^\nu)
        + (\partial^\nu \partial^\alpha A^\mu
        - \partial^\nu \partial^\alpha A^\mu)\\
        &\quad+ (\partial^\mu\partial^\nu A^\alpha
        - \partial^\mu\partial^\nu A^\alpha)\\
        &= 0
    \end{align*}
    Where we used that the order of partial derivatives does not matter.
\end{proof}
\cleardoublepage
\begin{proof}{\textbf{P7.1}}
    From the general equation for the transformation properties of a tensor
    we have that
    \begin{align*}
        F'^{\mu\nu} = \partialderivative{x'^\mu}{x^\alpha}
        \partialderivative{x'^\nu}{x^\beta} F^{\alpha\beta}
    \end{align*}
    In the special case for the Lorentz transformations the partial derivatives
    become
    \begin{align*}
        \partialderivative{x'^\mu}{x^\alpha} = \begin{bmatrix}
            \gamma & -\gamma\beta & 0 & 0\\
            -\gamma\beta & \gamma & 0 & 0\\
            0 & 0 & 1 & 0\\
            0 & 0 & 0 & 1
        \end{bmatrix}
    \end{align*}
    So in particular for $\mu=t$ and $\nu=x$ we get that
    \begin{align*}
        F'^{tx} = E'_x &= \partialderivative{x'^t}{x^\alpha}
        \partialderivative{x'^x}{x^\beta} F^{\alpha\beta}\\
        &= \partialderivative{x'^t}{x^\alpha}[
            \partialderivative{x'^x}{x^t} F^{\alpha t}
            + \partialderivative{x'^x}{x^x} F^{\alpha x}
            + \partialderivative{x'^x}{x^y} F^{\alpha y}
            + \partialderivative{x'^x}{x^z} F^{\alpha z}
        ]\\
        &= \partialderivative{x'^t}{x^\alpha}[
            -\gamma\beta F^{\alpha t}
            + \gamma F^{\alpha x}
        ]\\
        &= \partialderivative{x'^t}{x^t}[
            -\gamma\beta F^{t t}
            + \gamma F^{t x}
        ] + \partialderivative{x'^t}{x^x}[
            -\gamma\beta F^{x t}
            + \gamma F^{x x}
        ]\\
        &= \gamma^2 E_x - \gamma^2\beta^2 E_x\\
        &= \gamma^2 E_x(1 - \beta^2)\\
        &= E_x
    \end{align*}
    For $\mu=t$ and $\nu=y$ we get that
    \begin{align*}
        F'^{ty} = E'_y &= \partialderivative{x'^t}{x^\alpha}
        \partialderivative{x'^y}{x^\beta} F^{\alpha\beta}\\
        &= \partialderivative{x'^t}{x^\alpha}[
            \partialderivative{x'^y}{x^t} F^{\alpha t}
            + \partialderivative{x'^y}{x^x} F^{\alpha x}
            + \partialderivative{x'^y}{x^y} F^{\alpha y}
            + \partialderivative{x'^y}{x^z} F^{\alpha z}
        ]\\
        &= \partialderivative{x'^t}{x^\alpha}F^{\alpha y}\\
        &= \partialderivative{x'^t}{x^t}F^{t y}
        + \partialderivative{x'^t}{x^x}F^{x y}\\
        &= \gamma E_y - \gamma\beta B_z
    \end{align*}
    And for $\mu=t$ and $\nu=z$ we get that
    \begin{align*}
        F'^{tz} = E'_z &= \partialderivative{x'^t}{x^\alpha}
        \partialderivative{x'^z}{x^\beta} F^{\alpha\beta}\\
        &= \partialderivative{x'^t}{x^\alpha}[
            \partialderivative{x'^z}{x^t} F^{\alpha t}
            + \partialderivative{x'^z}{x^x} F^{\alpha x}
            + \partialderivative{x'^z}{x^y} F^{\alpha y}
            + \partialderivative{x'^z}{x^z} F^{\alpha z}
        ]\\
        &= \partialderivative{x'^t}{x^\alpha}F^{\alpha z}\\
        &= \partialderivative{x'^t}{x^t}F^{t z}
        + \partialderivative{x'^t}{x^x}F^{x z}\\
        &= \gamma E_z + \gamma\beta B_y
    \end{align*}
    On the other hand, for the magnetic field, taking $\mu=y$ and $\nu=z$
    we get the following
    \begin{align*}
        F'^{yz} = B'_x &= \partialderivative{x'^y}{x^\alpha}
        \partialderivative{x'^z}{x^\beta} F^{\alpha\beta}\\
        &= \partialderivative{x'^y}{x^\alpha}[
            \partialderivative{x'^z}{x^t} F^{\alpha t}
            + \partialderivative{x'^z}{x^x} F^{\alpha x}
            + \partialderivative{x'^z}{x^y} F^{\alpha y}
            + \partialderivative{x'^z}{x^z} F^{\alpha z}
        ]\\
        &= \partialderivative{x'^y}{x^\alpha}F^{\alpha z}\\
        &= \partialderivative{x'^y}{x^y}F^{y z}\\
        &= B_x
    \end{align*}
    For $\mu=z$ and $\nu=x$ we have that
    \begin{align*}
        F'^{zx} = B'_y &= \partialderivative{x'^z}{x^\alpha}
        \partialderivative{x'^x}{x^\beta} F^{\alpha\beta}\\
        &= \partialderivative{x'^z}{x^\alpha}[
            \partialderivative{x'^x}{x^t} F^{\alpha t}
            + \partialderivative{x'^x}{x^x} F^{\alpha x}
            + \partialderivative{x'^x}{x^y} F^{\alpha y}
            + \partialderivative{x'^x}{x^z} F^{\alpha z}
        ]\\
        &= \partialderivative{x'^z}{x^\alpha}[
            \partialderivative{x'^x}{x^t} F^{\alpha t}
            + \partialderivative{x'^x}{x^x} F^{\alpha x}
        ]\\
        &= \partialderivative{x'^z}{x^z}[
            \partialderivative{x'^x}{x^t} F^{z t}
            + \partialderivative{x'^x}{x^x} F^{z x}
        ]\\
        &= \gamma B_y + \gamma\beta E_z
    \end{align*}
    And finally, for $\mu=x$ and $\nu=y$ we have that
    \begin{align*}
        F'^{xy} = B'_z &= \partialderivative{x'^x}{x^\alpha}
        \partialderivative{x'^y}{x^\beta} F^{\alpha\beta}\\
        &= \partialderivative{x'^x}{x^\alpha}[
            \partialderivative{x'^y}{x^t} F^{\alpha t}
            + \partialderivative{x'^y}{x^x} F^{\alpha x}
            + \partialderivative{x'^y}{x^y} F^{\alpha y}
            + \partialderivative{x'^y}{x^z} F^{\alpha z}
        ]\\
        &= \partialderivative{x'^z}{x^\alpha} F^{\alpha y}\\
        &= \partialderivative{x'^x}{x^t}F^{t y}
        + \partialderivative{x'^x}{x^x}F^{x y}\\
        &= \gamma B_z - \gamma\beta E_y
    \end{align*}
\end{proof}
\cleardoublepage
\begin{proof}{\textbf{P7.2}}
\begin{itemize}
    \item [\textbf{a.}] The equation 7.20 states that
    $$\partial^\alpha F^{\mu\nu} + \partial^\nu F^{\alpha\mu}
    + \partial^\mu F^{\nu\alpha} = 0$$
    In the first term, for example, we know that $F^{\mu\nu}$ has 16 components
    since $F^{\mu\nu}$ is a second rank tensor.
    Viewing then $\partial^\alpha F^{\mu\nu}$ as a product of tensors of rank
    $1$ and rank $2$ the result gives us a tensor of rank $1 + 2 = 3$ which
    by definition has $64$ components.
    The same can be said for the rest of the terms so the equation has $64$
    components.

    \item[\textbf{b.}] Let us suppose that $\mu = \nu$ and $\alpha \neq \mu$
    then we see that
    \begin{align*}
        \partial^\alpha F^{\mu\mu} + \partial^\mu F^{\alpha\mu}
        + \partial^\mu F^{\mu\alpha} = 0\\
        0 + \partial^\mu F^{\alpha\mu}
        + \partial^\mu F^{\mu\alpha} = 0\\
        \partial^\mu F^{\alpha\mu}
        - \partial^\mu F^{\alpha\mu} = 0
    \end{align*}
    Where we used that $F^{\mu\mu} = 0$ and that
    $F^{\mu\alpha} = - F^{\alpha\mu}$ since $F^{\mu\nu}$ is antisymmetric.
    So in this case, we see that the equation is identically zero.

    In the case where $\mu = \nu = \alpha$ since $F^{\mu\mu} = 0$ then
    the equation is also identically zero.

    Therefore in the only case where the equation is not identically zero
    is when $\mu \neq \nu \neq \alpha$.

    \item[\textbf{c.}] Let $\alpha = t$, $\mu = x$ and $\nu = y$ then we have
    that
    \begin{align*}
        \partial^t F^{xy} + \partial^y F^{tx} + \partial^x F^{yt} &= 0\\
        \partial^t B_z + \partial^y E_x - \partial^x E_y &= 0\\
        -\partialderivative{B_z}{t} + \partialderivative{E_x}{y}
        - \partialderivative{E_y}{x} &= 0\\
        \partialderivative{B_z}{t} + \partialderivative{E_y}{x}
        - \partialderivative{E_x}{y}&= 0
    \end{align*}
    Where we see that $\partial^t B_z$ is the negative $z$ component of
    $\partial \vec{B} / \partial t$.

    On the other hand, let us compute $\curl \vec{E}$ as follows
    \begin{align*}
        \curl \vec{E} &= \begin{vmatrix}
            \hatx & \haty & \hatz \\
            \partialderivative{}{x} & \partialderivative{}{y}
            & \partialderivative{}{z}\\
            E_x  & E_y & E_z
        \end{vmatrix}\\
        &= \partialderivative{E_z}{y}\hatx
        + \partialderivative{E_x}{z}\haty
        + \partialderivative{E_y}{x}\hatz
        - \partialderivative{E_y}{z}\hatx
        - \partialderivative{E_z}{x}\haty
        - \partialderivative{E_x}{y}\hatz\\
        &= \bigg(\partialderivative{E_z}{y}- \partialderivative{E_y}{z}\bigg)\hatx
        + \bigg(\partialderivative{E_x}{z}
        - \partialderivative{E_z}{x}\bigg)\haty
        + \bigg(\partialderivative{E_y}{x}
        - \partialderivative{E_x}{y}\bigg)\hatz
    \end{align*}
    So we see that $\partial^y E_x - \partial^x E_y$ is the negative $z$
    component of $\curl \vec{E}$.
    Therefore with these indexes, we get the $z$ component of Faraday's law
    as shown.
\cleardoublepage
    Let now $\alpha = x$, $\mu = y$ and $\nu = z$
    \begin{align*}
        \partial^x F^{yz} + \partial^z F^{xy} + \partial^y F^{zx} &= 0\\
        \partial^x B_x + \partial^z B_z + \partial^y B_y &= 0\\
        \partialderivative{B_x}{x} + \partialderivative{B_y}{y}
        + \partialderivative{B_z}{z}&= 0
    \end{align*}
    Therefore we see that this is $\div \vec{B} = 0$ i.e. Gauss's law for
    the magnetic field.

    Finally for $\alpha = y$, $\mu = z$ and $\nu = t$ we get that
    \begin{align*}
        \partial^y F^{zt} + \partial^t F^{yz} + \partial^z F^{ty} &= 0\\
        -\partial^y E_z + \partial^t B_x + \partial^z E_y &= 0\\
        - \partialderivative{B_x}{t}
        + \partialderivative{E_y}{z}
        - \partialderivative{E_z}{y} &= 0\\
        \partialderivative{B_x}{t}
        + \partialderivative{E_z}{y}
        - \partialderivative{E_y}{z} &= 0
    \end{align*}
    Where we see that $\partial^t B_x$ is the negative $x$ component of
    $\partial \vec{B} / \partial t$ and $\partial^z E_y - \partial^y E_z$
    is the negative $x$ component of $\curl \vec{E}$.

    Therefore we get the $x$ component of Faraday's law with these indexes.
\end{itemize}
\end{proof}
\cleardoublepage
\begin{proof}{\textbf{P7.3}}
\begin{itemize}
    \item [\textbf{a.}] Equation (7.5) states the following
    \begin{align*}
        \partial_\nu F^{\mu\nu} = 4\pi k J^\mu
    \end{align*}
    So since $F^{\mu\nu} = \partial^\mu A^\nu - \partial^\nu A^\mu$ we have
    that
    \begin{align*}
        \partial_\nu (\partial^\mu A^\nu - \partial^\nu A^\mu) = 4\pi k J^\mu
    \end{align*}

    \item [\textbf{b.}] Let $A^\mu_{\text{new}} = A^\mu + \partial^\mu\Lambda$
    be a new four-potential then the equation (7.5) becomes
    \begin{align*}
        \partial_\nu (\partial^\mu A^\nu_{\text{new}} - \partial^\nu A^\mu_{\text{new}})
        &= 4\pi k J^\mu\\
        \partial_\nu (\partial^\mu (A^\nu + \partial^\nu\Lambda)
         - \partial^\nu (A^\mu + \partial^\mu \Lambda))
        &= 4\pi k J^\mu\\
        \partial_\nu (\partial^\mu A^\nu + \partial^\mu\partial^\nu\Lambda
         - \partial^\nu A^\mu - \partial^\mu\partial^\nu \Lambda)
        &= 4\pi k J^\mu\\
        \partial_\nu (\partial^\mu A^\nu - \partial^\nu A^\mu)
        &= 4\pi k J^\mu
    \end{align*}
    Where we used the fact that the order in which we apply the partial
    derivatives don't matter.
    Therefore we get the equation (7.5) in its original.

    \item [\textbf{c.}] Let $A^\mu$ now be a four-potential such that
    $\partial_\mu A^\mu = 0$ then the equation (7.5) becomes
    \begin{align*}
        \partial_\nu (\partial^\mu A^\nu - \partial^\nu A^\mu) &= 4\pi k J^\mu\\
        \partial^\mu \partial_\nu A^\nu - \partial_\nu \partial^\nu A^\mu &= 4\pi k J^\mu\\
        - \partial_\nu \partial^\nu A^\mu &= 4\pi k J^\mu\\
        \partial_\nu \partial^\nu A^\mu &= -4\pi k J^\mu
    \end{align*}
\end{itemize}
\end{proof}
\cleardoublepage
\begin{proof}{\textbf{P7.5}}
    We know from equation (7.7) that
    \begin{align*}
        \derivative{p^\mu}{\tau} = qF^{\mu\nu}u_\nu
    \end{align*}
    So for example for $\mu = x$ we have that
    \begin{align*}
        \derivative{p^x}{\tau} &= qF^{x\nu}u_\nu\\
        \derivative{p^x}{\tau} &=
        q(F^{xt}u_t + F^{xx}u_x + F^{xy}u_y + F^{xz}u_z)\\
        \derivative{p^x}{\tau} &= q(-E_xu_t + B_zu_y - B_yu_z)\\
        \derivative{p^x}{\tau} &= q(E_xu^t + B_zu^y - B_yu^z)
    \end{align*}
    But since the particle is moving relativistically we have that
    \begin{align*}
        \derivative{p^x}{\tau} &=
        q\bigg(\frac{E_x}{\sqrt{1 - v^2}} + B_z\frac{v_y}{\sqrt{1 - v^2}}
        - B_y\frac{v_z}{\sqrt{1 - v^2}}\bigg)
    \end{align*}
    And mutiplying the equation by $d\tau/dt$ we get that
    \begin{align*}
        \derivative{\tau}{t}\derivative{p^x}{\tau} &=
        q\bigg(\frac{E_x}{\sqrt{1 - v^2}} + B_z\frac{v_y}{\sqrt{1 - v^2}}
        - B_y\frac{v_z}{\sqrt{1 - v^2}}\bigg)\derivative{\tau}{t}\\
        \derivative{p^x}{t} &=
        q\bigg(\frac{E_x}{\sqrt{1 - v^2}} + B_z\frac{v_y}{\sqrt{1 - v^2}}
        - B_y\frac{v_z}{\sqrt{1 - v^2}}\bigg)\derivative{t\sqrt{1 - v^2}}{t}\\
        \derivative{p^x}{t} &= q(E_x + B_z v_y - B_y v_z)
    \end{align*}
    Which is the $x$ component of the Lorentz force equation in the reference
    frame where $t, \vec{v}, \vec{E}$ and $\vec{B}$ are measured.

    In the same way, for $\mu = y$ we have that
    \begin{align*}
        \derivative{p^y}{\tau} &= qF^{y\nu}u_\nu\\
        \derivative{p^y}{\tau} &=
        q(F^{yt}u_t + F^{yx}u_x + F^{yy}u_y + F^{yz}u_z)\\
        \derivative{p^y}{\tau} &= q(-E_yu_t - B_zu_x + B_xu_z)\\
        \derivative{p^y}{\tau} &= q(E_yu^t + B_xu^z - B_zu^x)\\
        \derivative{\tau}{t}\derivative{p^y}{\tau} &=
        q\bigg(\frac{E_y}{\sqrt{1 - v^2}} + B_x\frac{v_z}{\sqrt{1 - v^2}}
        - B_z\frac{v_x}{\sqrt{1 - v^2}}\bigg)\derivative{\tau}{t}\\
        \derivative{p^y}{t} &= q(E_y + B_x v_z - B_z v_x)
    \end{align*}
    And finally, for $\mu = z$ we have that
    \begin{align*}
        \derivative{p^z}{\tau} &= qF^{z\nu}u_\nu\\
        \derivative{p^z}{\tau} &=
        q(F^{zt}u_t + F^{zx}u_x + F^{zy}u_y + F^{zz}u_z)\\
        \derivative{p^z}{\tau} &= q(-E_zu_t + B_yu_x - B_xu_y)\\
        \derivative{p^z}{\tau} &= q(E_zu^t + B_yu^x - B_xu^y)\\
        \derivative{\tau}{t}\derivative{p^z}{\tau} &=
        q\bigg(\frac{E_z}{\sqrt{1 - v^2}} + B_y\frac{v_x}{\sqrt{1 - v^2}}
        - B_x\frac{v_y}{\sqrt{1 - v^2}}\bigg)\derivative{\tau}{t}\\
        \derivative{p^z}{t} &= q(E_z + B_y v_x - B_x v_y)
    \end{align*}
    Which are the $y$ and $z$ components of the Lorentz force equation.
    Therefore the Lorentz force equation is correct even in the relativistic
    limit.
\end{proof}
\end{document}